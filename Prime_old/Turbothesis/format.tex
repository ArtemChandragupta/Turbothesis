%%%%%%%%%%%%%%%%%% Шрифт и язык %%%%%%%%%%%%%%%%%%%%%%%%%%%%%%%%%

\setdefaultlanguage[spelling=modern]{russian}
\setotherlanguage{english}

\setmonofont{NotoMono Nerd Font}
\setmainfont{Times New Roman} 
\setromanfont{Times New Roman} 
\newfontfamily\cyrillicfont{Times New Roman}
\newfontfamily\cyrillicfontsf{Times New Roman}
\newfontfamily{\cyrillicfonttt}{JetBrainsMono Nerd Font}

\urlstyle{same} % шрифт для URL-ссылок

%%%%%%%%%%%%%%%%%% Геометрия %%%%%%%%%%%%%%%%%%%%%%%%%%%%%%%%%

\usepackage[
  left=3cm,
  right=1.5cm,
  top=2cm,
  bottom=2cm
]{geometry} % поля

% LaTeX, а конкретно - setspace, рассматривает межстрочный интервал не так, как MS Word:https://forum.typst.app/t/whats-the-equivalent-of-ms-words-1-5-line-spacing/1057
% \linespread{1.5} % междустрочный интервал
\linespread{1.25} % междустрочный интервал

\emergencystretch=25pt % Перенос текста при переполнении

\usepackage{indentfirst}       % пакет для отступа абзаца
\setlength{\parindent}{1.25cm} % отступ для абзаца

% Внесение titlepage в учёт счётчика страниц
\makeatletter
\renewenvironment{titlepage} {
  \thispagestyle{empty}
}

\usepackage{fancyhdr} % пакет для красивых полей снизу
\pagestyle{fancy}
\fancyhead{}
\fancyfoot{}
\fancyfoot[R]{\thepage} % номера страниц справа
\renewcommand{\headrulewidth}{0pt} % убираем линию сверху

%%%%%%%%%%%%%%%%%% Дополнения %%%%%%%%%%%%%%%%%%%%%%%%%%%%%%%%%

\graphicspath{ {./Images/} } % Путь до папки с изображениями

\addto\captionsrussian{
  \renewcommand{\contentsname}{СОДЕРЖАНИЕ}
}
\renewcommand{\cfttoctitlefont}{\normalfont\fontsize{14pt}{0}\bfseries}
\renewcommand{\l@section}{\@dottedtocline{1}{0.5em}{1.2em}} % Точки у секций, а не только субсекций в содержании.

\usepackage{titlesec}
\titleformat{\section}{\normalfont\fontsize{14pt}{0}\bfseries}{\thesection}{1.25em}{}
\titleformat{\subsection}{\normalfont\fontsize{14pt}{0}\bfseries}{\thesubsection}{1.25em}{}

\usepackage[figurename=Рисунок]{caption}
