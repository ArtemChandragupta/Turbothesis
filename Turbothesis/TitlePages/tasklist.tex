
\documentclass[12pt, a4paper]{extarticle}

\usepackage{polyglossia} % языковой пакет

\usepackage{pdfpages} % пакет для импорта pdf-файлов

\usepackage{tocvsec2} %%%%%%%%%%%%%%%%%%%%%%%%%%%%%%%%%

\usepackage{longtable,booktabs,array}

\usepackage{fontspec}

\usepackage{ulem}

\usepackage{setspace}

\usepackage[labelsep=period]{caption}

\usepackage{caption}

\usepackage{graphicx} % пакет для использования графики (чтобы вставлять рисунки, фотографии и пр.)
\usepackage{pgf}      % пакет для использование pgf

\usepackage{amsmath} % поддержка математических символов

\usepackage{url} % поддержка url-ссылок

\usepackage{csquotes}
\usepackage[ 
  backend=biber,
  sorting=none,
  bibstyle=gost-numeric,
  citestyle=gost-numeric
]{biblatex} % Библиография по ГОСТу

\addbibresource{refs.bib}

\usepackage{multirow} % таблицы с объединенными строками

\usepackage{hyperref} % пакет для интеграции гиперссылок

\usepackage{indentfirst} % пакет для отступа абзаца

\usepackage{chngcntr} % пакет подписей и нумерации рисунков

\usepackage{verbatim}
\usepackage{fancyvrb}

%%%%%%%%%%%%%%%%%%%%%%%%%%%%%%%%%%%%%%%%%%%%%%%%%%%%%%%%%%%%%%


%%%%%%%%%%%%%%%%%% Шрифт и язык %%%%%%%%%%%%%%%%%%%%%%%%%%%%%%%%%

\setdefaultlanguage[spelling=modern]{russian}
\setotherlanguage{english}
    
\setmonofont{NotoMono Nerd Font}
\setmainfont{Times New Roman} 
\setromanfont{Times New Roman} 
\newfontfamily\cyrillicfont{Times New Roman}
\newfontfamily\cyrillicfontsf{Times New Roman}
\newfontfamily{\cyrillicfonttt}{JetBrainsMono Nerd Font}

\urlstyle{same} % шрифт для URL-ссылок

%%%%%%%%%%%%%%%%%% Геометрия %%%%%%%%%%%%%%%%%%%%%%%%%%%%%%%%%

\usepackage[
  left=3cm,
  right=1.5cm,
  top=2cm,
  bottom=2cm
]{geometry} % поля

\linespread{1.5} % междустрочный интервал

\emergencystretch=25pt % Перенос текста при переполнении

\usepackage{indentfirst}       % пакет для отступа абзаца
\setlength{\parindent}{1.25cm} % отступ для абзаца

% Внесение titlepage в учёт счётчика страниц
\makeatletter
\renewenvironment{titlepage} {
  \thispagestyle{empty}
}

\usepackage{fancyhdr} % пакет для красивых полей снизу
\pagestyle{fancy}
\fancyhead{}
\fancyfoot{}
\fancyfoot[R]{\thepage} % номера страниц справа
\renewcommand{\headrulewidth}{0pt} % убираем линию сверху

%%%%%%%%%%%%%%%%%% Дополнения %%%%%%%%%%%%%%%%%%%%%%%%%%%%%%%%%

\graphicspath{ {./Images/} } % Путь до папки с изображениями

\renewcommand{\l@section}{\@dottedtocline{1}{0.5em}{1.2em}} % Точки у секций, а не только субсекций в содержании.

% Цвет гиперссылок и цитирования
% \usepackage{hyperref}
% \hypersetup{
%   colorlinks=true, 
%   linkcolor=black, 
%   filecolor=blue, 
%   citecolor = black,       
%   urlcolor=blue, 
% }

\counterwithin{figure}{section} % Нумерация рисунков
\counterwithin{table}{section} % Нумерация таблиц


\renewcommand{\ULdepth}{1.8pt}

\usepackage[document]{ragged2e}

% для данного титульника слева оступ 3см, справа 1 см, сверху 2см, снизу 0
\usepackage[left=3cm,right=1cm,top=2cm,bottom=0cm]{geometry}

\begin{document}

\fontsize{12pt}{1.08}\selectfont

\begin{center}
\includegraphics[width=0.97917in,height=1.10417in]{Images/mirea.png}

\setlength{\parskip}{6pt}
МИНОБРНАУКИ РОССИИ 

Федеральное государственное бюджетное образовательное учреждение
\setlength{\parskip}{0pt}

высшего образования

\textbf{«МИРЭА -- Российский технологический университет»}

\fontsize{16pt}{1.08}\selectfont
\textbf{РТУ МИРЭА}
\fontsize{12pt}{1.08}\selectfont

\setlength{\parskip}{6pt}
\noindent\rule{\textwidth}{2pt}

\textbf{Институт информационных технологий (ИТ)}
\setlength{\parskip}{0pt}

\textbf{Кафедра инструментального и прикладного программного обеспечения (ИиППО)}

\bigskip
\textbf{ЗАДАНИЕ}

\textbf{на выполнение курсовой работы}

\end{center}

\justifying

\setlength{\parindent}{0in}
по дисциплине: \uline{Разработка клиентских частей
интернет-ресурсов}

по профилю: \uline{Разработка программных продуктов и проектирование
информационных систем}

направления профессиональной подготовки: \uline{Программная
инженерия (09.03.04)}

\medskip

Студент: \uline{Иванов Иван Иванович}

Группа: \uline{ИКБО-13-19}

Срок представления к защите: \uline{00.12.2020}

Руководитель: \uline{Иванов Иван Иванович, к.т.н., доцент}

\bigskip

\textbf{Тема:} «\uline{Интернет-ресурс на тему «История анимации» с
применением технологий HTML5, CSS3, JavaScript»}

\bigskip

\textbf{Исходные данные:} \uline{используемые технологии: HTML5,
CSS3, JavaScript, текстовый редактор Notepad++/Visual Studio Code/Atom
(на выбор), наличие: интерактивного поведения веб-страниц, межстраничной
навигации, внешнего вида страниц, соответствующего современным
стандартам веб-разработки; инструменты и технологии адаптивной верстки
для полноценного отображения контента на различных браузерах и видах
устройств. Нормативный документ: инструкция по организации и проведению
курсового проектирования СМКО МИРЭА 7.5.1/04.И.05-18.}

\bigskip

\textbf{Перечень вопросов, подлежащих разработке, и обязательного
графического материала:} \uline{1. Провести анализ предметной
области разрабатываемого интернет-ресурса. 2. Обосновать выбор
технологий разработки интернет-ресурса. 3. Создать пять и более
веб-страниц интернет-\\ресурса с использованием технологий HTML5, CSS3 и
JavaScript.\\
4. Организовать межстраничную навигацию. 5. Реализовать слой клиентской
логики веб-страниц с применением технологии JavaScript. 6. Провести
оптимизацию веб-страниц и размещаемого контента для браузеров и
различных видов устройств. 7. Создать презентацию по выполненной
курсовой работе.}

\bigskip

Руководителем произведён инструктаж по технике безопасности,
противопожарной технике и правилам внутреннего распорядка.

\fontsize{12pt}{1.5}\selectfont

\bigskip

Зав. кафедрой ИиППО: \_\_\_\_\_\_\_\_\_\_\_/Р. Г. Болбаков/,
«\_\_\_\_\_»\_\_\_\_\_\_\_\_\_\_\_\_2020 г.

\medskip

Задание на КР выдал: \_\_\_\_\_\_\_\_\_\_\_\_\_\_\_/А.А Иванов/,
«\_\_\_\_\_»\_\_\_\_\_\_\_\_\_\_\_\_2020 г.

\medskip

Задание на КР получил: \_\_\_\_\_\_\_\_\_\_\_/И.И. Иванов/,
«\_\_\_\_\_»\_\_\_\_\_\_\_\_\_\_\_\_2020 г.

\end{document}
