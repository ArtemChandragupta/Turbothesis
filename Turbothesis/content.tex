\section{Введение}
  Принцип действия ГТУ сводится к следующему. Из атмосферы воздух забирают компрессором, после чего при повышенном давлении его подают в камеру сгорания, куда одновременно подводят жидкое топливо топливным насосом или газообразное топливо от газового компрессора. В камере сгорания воздух разделяется на два потока: один поток в количестве, необходимом для сгорания топлива, поступает внутрь жаровой трубы; второй – обтекает жаровую трубу снаружи и подмешивается к продуктам сгорания для понижения их температуры. Процесс сгорания в камере происходит при почти постоянном давлении.   Получающийся после смешения газ поступает в газовую турбину, в которой, расширяясь, совершает работу, а затем выбрасывается в атмосферу.   В отличие от паротурбинной установки полезная мощность ГТУ составляет только 30-50\% мощности турбины. Долю полезной мощности можно увеличить, повысив температуру газа перед турбиной или снизить температуру воздуха, засасываемого компрессором. В первом случае возрастает работа расширения газа в турбине, во втором – уменьшается работа, затрачиваемая на сжатие воздуха в компрессоре. Оба способа приводят к увеличению доли полезной мощности. Полезная мощность ГТУ также зависит от аэродинамических показателей проточных частей турбины и компрессора: чем меньше аэродинамические потери в турбине и компрессоре, тем большая доля мощности газовой турбины становится полезной.  Эффективность ГТУ в сравнении с другими тепловыми двигателями обнаруживается только при высокой температуре газа и высокой экономичности турбины и компрессора. Поэтому простой по принципу действия газотурбинный двигатель стали применяется в промышленности позднее других тепловых двигателей, т.е после того, как был достигнут прогресс в технологии получения жаропрочных материалов и накоплены необходимые знания в области аэродинамики турбомашин.

\newpage
\section{Термодинамический и газодинамический расчет}
\subsection{Исходные данные}

\begin{enumerate} 
  \item Полезная мощность $N=\N \ \text{МВт}$
  \item Температура газа перед турбиной $T_3^*=\Ttri\ \text{К}$
  \item Параметры наружного воздуха $p_H=\PN \ \text{МПа}; T_H=\TN \ \text{К}$
  \item Топливо – природный газ
  \item Прототип установки – ГТЭ-65
  \item Частота вращения вала - $n=3000 \ \dfrac{\text{об}}{\text{мин}}$
\end{enumerate} 
Примем два упрощения при расчете в разделе 2:
\begin{enumerate} 
  \item Охлаждение турбины не учитывается, расход охладителя равен нулю;
  \item Не учитывается зависимость теплоемкости газа от температуры рабочего тела, принимается по рекомендациям пособия \cite{PERV}.
\end{enumerate}

  \image{HeatScheme.jpg}{Тепловая схема одновальной ГТУ: К-компрессор; КС-камера сгорания; ГТ-газовая турбина.}{0.7}

  \image{HeatGraph.png}{Цикл одновальной ГТУ простого типа в T-s–диаграмме: 1-2 – адиабатное сжатие в компрессоре, 2-3 – изобарный подвод теплоты в камере сгорания, 3-4 – адиабатное расширение продуктов сгорания на лопатках газовой турбины, 4-1 – изобарный отвод теплоты от продуктов сгорания в атмосферу.}{0.7}

\subsection{Методы и пример расчета параметров рабочего процесса в характерных сечениях проточной части ГТУ. Определение основных характеристик ГТУ}

Расчет производится по методике из пособия \cite{PERV} (с.77-78)

Зададимся параметром степени повышения давления $\pi_{\text{к}}^*=\dfrac{P_{2}^*}{P_{1}^*}=\Pik$. \\
Газовая постоянная воздуха: $R_{\text{в}}=\RN \ \dfrac{\text{кДж}}{\text{кг}\cdot \text{К}}$ \\
Удельная изобарная теплоёмкость воздуха: $c_{p_\text{в}}=\CPN \ \dfrac{\text{кДж}}{\text{кг}\cdot \text{К}}$\\
Коэффициент Пуассона воздуха: $k_{\text{в}}=\dfrac{c_{p_{\text{в}}}}{c_{p_{\text{в}}}-R_{\text{в}}}=\dfrac{\CPN}{\CPN-\RN}=1.386$\\

Принимаем коэффициент потерь полного давления во входном устройстве ГТУ $\sigma^*_{\text{вх}}=\SigEN$;

Давление воздуха перед компрессором:
\begin{equation} \label{eu_eqn}
	P_{1}^*=\sigma_{\text{вх}}^*\cdot P_{\text{н}}=\SigEN \cdot \PN=\Podi \ \text{МПа};
\end{equation}

Температура воздуха перед компрессором:
\begin{equation} \label{eu_eqn}
	T_{1}^*=T_{\text{н}}=\TN \ \text{К};
\end{equation}

Давление воздуха за компрессором:
\begin{equation} \label{eu_eqn}
	P_{2}^*=\pi_{\text{к}}^*\cdot P_{1}^*=\Podi \cdot \Pik = \Pdwa \ \text{МПа};
\end{equation}

Определим $T_2^*$ (температуру воздуха за компрессором):
\begin{equation} \label{eu_eqn}
	T_{2}^*=T_{\text{н}}^*\cdot(\pi_{\text{к}}^*)^{\frac{k_{\text{в}}-1}{k_{\text{в}}}}=\TN \cdot \Pik^{0.279}=\Tdwa \ \text{К};
\end{equation}

Работа, соответствующая изоэнтропийному перепаду в компрессоре:
\begin{equation} \label{eu_eqn}
H_{\text{ок}}^*=C_{p_{\text{в}}}\cdot T_{1}^*\cdot\left[ (\pi_{\text{к}}^*)^{\frac{k_{\text{в}}-1}{k_{\text{в}}}}-1 \right]=\CPN \cdot \TN \cdot (\Pik ^{0.279}-1)=\Hok \ \dfrac{\text{кДж}}{\text{кг}};
\end{equation}

$\eta_{\text{к ад}}=\EtaKAD$ - политропный к.п.д. компрессора, его выбор для расчета обусловлен тем, что он мало зависит от степени повышения давления в компрессоре $\pi^*_{\text{к}}$.

Полезная работа в компрессоре:
\begin{equation} \label{eu_eqn}
	H_{\text{к}}=\dfrac{H_{\text{ок}}^*}{\eta_{\text{к ад}}}=\dfrac{\Hok}{\EtaKAD}=\Hk \  \dfrac{\text{кДж}}{\text{кг}};
\end{equation}

Принимаем коэффициент потерь полного давления в камере сгорания $\sigma^*_{\text{кс}}=\SigBU$;

Давление газа перед турбиной:
\begin{equation} \label{eu_eqn}
	p_{3}^*=p_{2}^* \cdot \sigma^*_{\text{кс}}= \pi_{\text{к}}^* \cdot p_{1}^* \cdot \sigma^*_{\text{кс}}= \Pik \cdot \Podi \cdot \SigBU=\Ptri \ \text{МПа};
\end{equation}

Принимаем коэффициент потерь полного давления в выходном устройстве ГТУ $\sigma^*_{\text{вых}}=\SigOUT$;

Давление газа за турбиной:
\begin{equation} \label{eu_eqn}
	p_{4}^*=\dfrac{p^*_{\text{н}}}{\sigma^*_{\text{вых}}}=\dfrac{\PN}{\SigOUT}= \PN \ \text{МПа};
\end{equation}

Степень расширения газа в турбине:
\begin{equation} \label{eu_eqn}
	\pi_{\text{т}}^*=\dfrac{p_{3}^*}{p_{4}^*}=\dfrac{\Ptri}{\Pdwa}=\PiT;
\end{equation}

$k_{\text{г}}= \Kg$ - показатель изоэнтропы газа;

$R_{\text{г}}= \RN \dfrac{\text{кДж}}{\text{кг}\cdot \ \text{К}}$;

$C_{p_{\text{в}}}=\CPg \dfrac{\text{кДж}}{\text{кг}\cdot \ \text{К}}$ - удельная изобарная теплоёмкость газа;

Работа, соответствующая изоэнтропийному перепаду в турбине:
\begin{equation} \label{eu_eqn}
	H_{\text{от}}^*=c_{p_{\text{г}}}\cdot T_{3}^*\cdot[1-(\pi_{\text{т}}^*)^{-\frac{k_{\text{г}}-1}{k_{\text{г}}}}]= \CPg \cdot \Ttri\cdot (1-(\PiT)^{-0.248})=\Hot \  \dfrac{\text{кДж}}{\text{кг}};
\end{equation}

Принимаем политропный кпд турбины $\eta_{\text{т пол}}=\EtaTpol$

Полезная работа в турбине:
\begin{equation} \label{eu_eqn}
	H_{\text{т}}=H_{\text{от}}^* \cdot \eta_{\text{т пол}}=\Hot \cdot \EtaTpol =\Ht \  \dfrac{\text{кДж}}{\text{кг}};
\end{equation}

Температура газа за турбиной $T_{4}^*$ определяется как:
\begin{equation} \label{eu_eqn}
	T_{4}^*=T_{3}^* \cdot (\pi_{\text{т}}^*)^{-\frac{k_{\text{г}}-1}{k_{\text{г}}}}=\Ttri \cdot (\PiT)^{-0.248}=\Tche \ \text{К};
\end{equation}

Примем коэффициенты механических потерь в турбине и компрессоре $\eta_{\text{мт}}=\Etamt$; $\eta_{\text{мк}}=\Etamk$;

Расход воздуха через компрессор:
\begin{equation} \label{eu_eqn}
	G_{\text{в}}=\dfrac{N_{e}}{H_{\text{т}}\cdot \eta_{\text{мт}}-\frac{H_{\text{к}}}{\eta_{\text{мк}}}}=\dfrac{\N \cdot 10^6}{\Ht \cdot 10^3 \cdot \Etamt -  \dfrac{\Hk \cdot 10^3}{\Etamk}}= \GN \ \dfrac{\text{кг}}{\text{с}};
\end{equation}

Теплоты с учетом потерь в камере сгорания:
\begin{equation} \label{eu_eqn}
	Q_{1}'=c_{p_\text{г}}\cdot(T_{3}^*-T_{2}^*)=1.16\cdot(1643.15-644.41)=1159 \ \dfrac{\text{кДж}}{\text{кг}};
\end{equation}

Примем КПД камеры сгорания $\eta_{\text{кс}}=\Etaks$;

Расход теплоты:
\begin{equation} \label{eu_eqn}
	Q_{1}=\dfrac{Q_{1}'}{\eta_{\text{кс}}}=\dfrac{\Qii}{\Etaks}=\Qodi \ \dfrac{\text{кДж}}{\text{кг}};
\end{equation}

Эффективный к.п.д. ГТУ:
\begin{equation} \label{eu_eqn}
	\eta_{\text{е}}=\dfrac{H_{\text{т}}\cdot \eta_{\text{мт}}-\frac{H_{\text{к}}}{\eta_{\text{мк}}}}{Q_{1}}=\dfrac{834.8\cdot 10^3 \cdot 0.995 -  \frac{403.4\cdot 10^3}{0.995}}{1182\cdot 10^3}=0.36;
\end{equation}

Коэффициент полезной работы ГТУ:
\begin{equation} \label{eu_eqn}
	\varphi=\dfrac{H_{\text{т}}\cdot \eta_{\text{мт}}-\frac{H_{\text{к}}}{\eta_{\text{мк}}}}{H_{\text{т}}\cdot \eta_{\text{мт}}}=\dfrac{\Ht \cdot 10^3 \cdot \Etamt- \frac{\Hk \cdot 10^3}{\Etamk}}{\Ht \cdot 10^3 \cdot \Etamt}= \phi;
\end{equation}

\newpage
\section{Вариантный расчет ГТУ на ЭВМ}

Необходимо провести расчет параметров рабочего процесса в характерных сечениях проточной части и основных характеристик ГТУ при различных значениях степени повышения давления и температуры газа перед турбиной, по результатам расчета построить графики: $H_{e}, \eta_{e}, \varphi=f(\pi_{\text{К}}^*, T_3^*)$
\subsection{Результаты расчета}

Полные результаты рассчета смотреть в Приложении 1.

\image{KPD.png}{Зависимость эффективного КПД ГТУ от степени повышения давления в компрессоре, при различных значениях температуры.}{1}

\image{Fi.png}{Зависимость коэффициента полезной работы ГТУ от степени повышения давления в компрессоре, при различных значениях температуры.}{1}

\image{He.png}{Зависимость эффективной удельной работы ГТУ от степени повышения давления в компрессоре, при различных значениях температуры.}{1}

\newpage
\subsection{Выбор степени повышения давления в компрессоре и начальной температуры газа перед турбиной}

	Максимальный КПД установки достигается при максимальной температуре газа перед турбиной – $1743 \text{к}$. Жаростойкость материала лопаток турбины не позволяет выдерживать такую температуру, поэтому в качестве входной температуры на турбину выберем $1693 \ \text{К}$. Экстремум графика зависимости эффективного КПД ГТУ от степени повышения давления в компрессоре наблюдается при $\Pi_{\text{к}}^*=34$ и $\eta_e = 0.419$. Выбор такой степени сжатия неоправдан, т. к. при нём слишком низкие значения эффективной удельной работы и коэффициента полезной работы. Экстремум графика зависимости эффективной удельной работы ГТУ от степени повышения давления в компрессоре наблюдается при $\Pi_{\text{к}}^*=16$, значение эффективного КПД ГТУ при этом $\eta_e = 0.314$. Коэффициент полезной работы ГТУ с увеличением  монотонно уменьшается, однако уменьшение  с целью увеличения  нецелесообразно, поскольку величина коэффициента полезной работы ГТУ увеличивается незначительно, при этом снижается величина эффективной удельной работы.

 Таким образом, для дальнейших расчетов принимаем:
 
$T_3^* = 1693 \text{К}; \ \Pi_{\text{к}}^*=16$

\newpage
\section{Приближённый рассчет осевого компрессора}

Расчет производится в соответствии со схематическим продольным разрезом на рис.4.1. по методике из пособия \cite{COMP}.

\image{COMP-full.png}{Схема многоступенчатого осевого компрессора.}{1}

При приближенном расчете осевого компрессора основными расчетными сечениями являются: сечение 1-1 на входе в первую ступень и сечение 2-2  на выходе из последней ступени (рис.4.2). Определим параметры $P$ и $T$ в трех сечениях.

Давление воздуха в сечении 1-1:
\begin{equation} \label{eu_eqn}
	p_{1}^*=\sigma_{\text{вх}}\cdot p_{\text{н}}=\Sigin \cdot \PN=\PCone \ \text{МПа};
\end{equation}

где коэффициент уменьшения полного давления во входной части компрессора $\sigma_{\text{вх}}^*=\Sigin$.

Температура в сечении 1-1:
\begin{equation} \label{eu_eqn}
	T_{1}^*=T_{\text{н}}=\TN \ \text{К};
\end{equation}

Давление воздуха в сечении К-К:
\begin{equation} \label{eu_eqn}
	p_{\text{к}}^*=p_{\text{н}}\cdot \pi_{\text{к}}^*=\PN \cdot \PiPi=\Pk \ \text{МПа};
\end{equation}

где $\Pi_{\text{к}}^*=16$ - степень повышения давления компрессоре.

Давление в сечении 2-2:
\begin{equation} \label{eu_eqn}
	p_{2}^*=\dfrac{p_{\text{к}}^*}{\sigma_{\text{вых}}^*}=\dfrac{\Pk}{\Sigout}=\PCtwo \  \text{МПа};
\end{equation}

где коэффициент увеличения давления в выходной части компрессора $\sigma_{\text{вых}}^*=\Sigout$.

\image{COMP-lopatki.png}{Схема первой и последней ступеней компрессора.}{0.7}

Значение плотностей:
\begin{equation} \label{eu_eqn}
	\rho_{1}=\dfrac{p_{1}^*}{R_{\text{в}}\cdot T_{1}^*}= \dfrac{\PCone}{\RN \cdot \TN} =\Rhoone \ \dfrac{\text{кг}}{\text{м}^3};
\end{equation}

Газовая постоянная воздуха: $R_{\text{в}}=\RN \ \dfrac{\text{кДж}}{\text{кг}\cdot k}$

Удельная изобарная теплоёмкость воздуха: $c_{p_\text{в}}=\CPN \ \dfrac{\text{кДж}}{\text{кг}\cdot k}$

\vspace{0.1cm}

Коэффициент Пуассона воздуха: $k_{\text{в}}=\dfrac{c_{p_{\text{в}}}}{c_{p_{\text{в}}}-R_{\text{в}}}=\dfrac{\CPN}{\CPN-\RN}=\KN$

Коэффициент полезного действия компресора: $\eta^*_{\text{ад}}=\Etad$

\begin{equation} \label{eu_eqn}
	\rho_{2}=\rho_{1}(\dfrac{p_{2}^*}{p_{1}^*})^{\frac{1}{n}}=\Rhoone(\dfrac{\PCtwo}{\PCone})^{\frac{1}{\nk}}=\Rhotwo \ \dfrac{\text{кг}}{\text{м}^3}
\end{equation}

где $n$ - показатель политропы определяется из равенства:
\begin{equation} \label{eu_eqn}
	\dfrac{k}{k-1}\cdot \eta_{\text{ад}}^*=\dfrac{n}{n-1};
\end{equation}
$$\dfrac{\KN}{\KN-1} \cdot \Etad=\dfrac{n}{n-1};$$
$$n = \nk.$$

Примем величины осевой составляющей абсолютных скоростей в сечении 1-1 и 2-2 $C_{z_1}=\Czone \  \dfrac{\text{м}}{\text{с}}$ и $C_{z_2}=\Cztwo \  \dfrac{\text{м}}{\text{с}}$. Втулочное отношение выберем $V_{1}=\dfrac{D_{\text{вт}_{1}}}{D_{\text{н}_{1}}}=\Vone$. Расход воздуха $G_{\text{в}}=\GNN \ \dfrac{\text{кг}}{\text{с}}$.

Из уравнения расхода первой ступени выразим значение наружного диаметра на входе в компрессор:
\begin{equation} \label{eu_eqn}
	G_{\text{в}}=\rho_{1}\cdot \dfrac{\pi}{4}\cdot(D_{\text{н}_{1}}^2\cdot D^2_{\text{вт}_{1}})\cdot C_{z_{1}}=\rho_{1}\cdot D_{\text{н}_{1}}\cdot(1-\nu_{1}^2)\cdot C_{z_{1}};
\end{equation}

Откуда,
$$D_{\text{н}_{1}}=\sqrt{ \dfrac{4\cdot G_{\text{в}}}{\rho_{1}\cdot \pi\cdot(1-\nu_1^2)\cdot C_{z_{1}}} }=\sqrt{ \dfrac{4\cdot \GNN}{\Rhoone \cdot 3.14\cdot (1-\Vone^2)\cdot \Czone} }=\Dodi \ \text{м};$$

Диаметр втулки первой ступени:
\begin{equation} \label{eu_eqn}
	D_{\text{вт}_{1}}=\nu_{1}\cdot D_{\text{н}_{1}}=\Vone \cdot \Dodi= \Dvtone \ \text{м};
\end{equation}

Средний диаметр первой ступени:
\begin{equation} \label{eu_eqn}
	D_{\text{ср}_{1}}=\dfrac{D_{\text{н}_{1}}+D_{\text{вт}_{1}}}{2}=\dfrac{\Dodi + \Dvtone}{2}=\Dsrone \ \text{м};
\end{equation}

Длина рабочей лопатки первой ступени:
\begin{equation} \label{eu_eqn}
	l_{1}=\dfrac{D_{\text{н}_{1}}-D_{\text{вт}_{1}}}{2}=\dfrac{\Dodi-\Dvtone}{2}=\Lone \ \text{м};
\end{equation}

Размеры проходного сечения 2-2:
\begin{equation} \label{eu_eqn}
	F_{2}=\dfrac{G_{\text{в}}}{C_{z_{2}}\cdot \rho_{2}}=\dfrac{\GNN}{\Cztwo \cdot \Rhotwo}=\Ftwo \ \text{м}^2;
\end{equation}

Принимаем в проточной части $D_{\text{вт}}=const$.

Тогда:
\begin{equation} \label{eu_eqn}
	\nu_{2}=\dfrac{1}{\sqrt{ \dfrac{1+4F_{2}}{\pi\cdot D_{\text{вт}_{1}}^2} }}=\dfrac{1}{\sqrt{ \dfrac{1+4\cdot \Ftwo}{\pi\cdot \Dvtone^2} }}=\Vtwo;
\end{equation}

Длина рабочей лопатки на последней ступени:
\begin{equation} \label{eu_eqn}
	l_{2}=\dfrac{1}{2}(\dfrac{1}{\nu_{2}}-1)\cdot D_{\text{вт}_{1}}=\dfrac{1}{2}(\dfrac{1}{\Vtwo}-1)\cdot \Dvtone=\Ltwo \ \text{м};
\end{equation}

Для расчета частоты вращения необходимо задать окружную скорость на наружном диаметре первой ступени $U_{\text{н}_{1}}=\Unone \ \dfrac{\text{м}}{\text{с}}$, тогда:
\begin{equation} \label{eu_eqn}
	n=\dfrac{60\cdot U_{н_{1}}}{\pi\cdot D_{\text{н}_{1}}}=\dfrac{60\cdot \Unone}{3.14\cdot \Dodi}=\n \approx 3000 \ \dfrac{\text{об}}{\text{мин}};
\end{equation}

Адиабатический напор в проточной части компрессора по полным параметрам:
\begin{multline} \label{eu_eqn}
	H_{\text{ад. пр. ч.}}^*=\dfrac{k}{k-1}\cdot R_{\text{в}}\cdot T_{1}^*\cdot \Bigr((\dfrac{p_{2}^*}{p_{1}^*})^{\frac{{k-1}}{k}}-1\Bigl)=\\
	=\dfrac{\KN}{\KN-1}\cdot \RN \cdot \TN \cdot \Bigr((\dfrac{\PCtwo}{\PCone})^{\frac{{\KN-1}}{\KN}}-1\Bigl)=\Ha \ \dfrac{\text{кДж}}{\text{кг}};
\end{multline}

Приближенная величина теоретического напора или удельная работа, затрачиваемая на сжатие 1 кг воздуха:
\begin{equation} \label{eu_eqn}
	H_{\text{к}}^*=\dfrac{H_{\text{ад. пр. ч.}}^*}{\eta_{\text{ад}^*}}=\dfrac{\Ha}{\Etad}=\Hkk \ \dfrac{\text{кДж}}{\text{кг}};
\end{equation}

Выберем средний теоретический напор $h_{\text{ср}}=\hcp \ \dfrac{\text{кДж}}{\text{кг}}$.

Число ступеней компрессора:
\begin{equation} \label{eu_eqn}
	i= \dfrac{H^*_{\text{к}}}{h_{\text{ср}}}=\dfrac{\Hkk}{\hcp}=14.77;
\end{equation}

Принимаем $i=\i$.

Теоретический напор в первой ступени:
\begin{equation} \label{eu_eqn}
	h_{1}=(0.6\dots 0.7)\cdot h_{\text{ср}}=0.65\cdot \hcp=\hone \ \dfrac{\text{кДж}}{\text{кг}};
\end{equation}

Теоретический напор в средних ступенях:
\begin{equation} \label{eu_eqn}
	h_{\text{ср. ст.}}=(1.1\dots 1.2)\cdot h_{\text{ср}}=1.15 \cdot \hcp=\hm \ \dfrac{\text{кДж}}{\text{кг}};
\end{equation}

Теоретический напор в последней ступени:
\begin{equation} \label{eu_eqn}
	h_{п}=(0.95\dots 1)\cdot h_{\text{ср}}=1\cdot \hcp=\hcp \ \dfrac{\text{кДж}}{\text{кг}};
\end{equation}

\image{COMP-stup.png}{Распределение теоретического напора по ступеням компрессора.}{1}

В результате распределения напоров соблюдается условие:

\begin{equation} \label{eu_eqn}
	\sum h_{i} \approx H^*_{\text{к}}=\Hkk \ \dfrac{\text{кДж}}{\text{кг}};
\end{equation}

Уточняем величину окружной скорости на среднем диаметре первой ступени:
\begin{equation} \label{eu_eqn}
	u_{\text{ср}_{1}}=\dfrac{\pi\cdot D_{\text{ср}_{1}}\cdot n}{60}=\dfrac{3.14\cdot \Dsrone \cdot 3000}{60}=\Usrone \ \dfrac{\text{м}}{\text{с}};
\end{equation}

Производим расчет первой ступени по среднему диаметру;

Коэффициент расхода на среднем диаметре:
\begin{equation} \label{eu_eqn}
	\varphi_{1}=\dfrac{C_{z_{1}}}{u_{\text{ср}_{1}}}=\dfrac{\Czone}{\Usrone}=\Phione;
\end{equation}

Коэффициент теоретического напора:
\begin{equation} \label{eu_eqn}
	\bar{h}_{1}=\dfrac{h_{1}}{u_{\text{ср}_{1}^2}}=\dfrac{\hone \cdot 10^3}{\Usrone^2}=\hhone;
\end{equation}

Отношение:
\begin{equation} \label{eu_eqn}
	\dfrac{\bar{h}_{1}}{\varphi}=\dfrac{\hhone}{\Phione}=\otn;
\end{equation}

Зададим степень реактивности $\Omega=\OME$ и найдем:
\begin{equation} \label{eu_eqn}
	\dfrac{\Omega}{\varphi}=\dfrac{\OME}{\Phione}=\otm;
\end{equation}

По графику 4.4 находим $\Bigl(\dfrac{\bar{h}_{1}}{\varphi} \Bigr)_{\frac{b}{t}=1}=\Pograph$.

\image{COMP-ome-approx.png}{График зависимости $\Bigl(\dfrac{\bar{h}_{1}}{\varphi} \Bigr)_{\frac{b}{t}=1}$ от $\dfrac{\Omega}{\varphi}$}{1}

Коэффициент:
\begin{equation} \label{eu_eqn}
	J= \dfrac{\dfrac{\bar{h}_{1}}{\varphi}}{\Bigl(\dfrac{\bar{h}_{1}}{\varphi} \Bigr)_{\frac{b}{t}=1}}= \dfrac{\otn}{\Pograph}=\J;
\end{equation}

\image{COMP-J.png}{Изменение коэффициента $J$ в зависимости от густоты решетки.}{1}

Пользуясь рисунком 4.5, определяем $\dfrac{b}{t}=\GraphJ \to \dfrac{t}{b}=\tb$.

При постоянной вдоль радиуса хорде относительный шаг у втулки первой ступени:
\begin{equation} \label{eu_eqn}
	\Bigl(\dfrac{t}{b}\Bigr)_{\text{вт}}=\dfrac{t}{b}\cdot \dfrac{D_{\text{вт}_{1}}}{D_{\text{ср}_{1}}}=\tb \cdot \dfrac{\Dvtone}{\Dsrone}=\tbem;
\end{equation}

Окружные скорости на входе и на выходе из рабочего колеса принимаем одинаковыми, т. е. $u_{\text{ср}_{1}}=u_{\text{ср}_{2}}=u=\U \ \dfrac{\text{м}}{\text{с}}$.

Проекция абсолютной скорости на окружное направление входной скорости на входе в рабочее колесо:
\begin{equation} \label{eu_eqn}
	C_{u_{1}}=u(1-\Omega)-\dfrac{h_{1}}{2u}=\U \cdot0.5- \dfrac{\hone \cdot 10^3}{2\cdot \U}=\Cuone\ \dfrac{\text{м}}{\text{с}};
\end{equation}

На выходе из рабочего колеса:
\begin{equation} \label{eu_eqn}
	C_{u_{2}}=u(1-\Omega)+\dfrac{h_{1}}{2u}=\U \cdot0.5+ \dfrac{\hone \cdot 10^3}{2\cdot \U}=\Cutwo \ \dfrac{\text{м}}{\text{с}};
\end{equation}

Абсолютная скорость на входе в рабочее колесо:
\begin{equation} \label{eu_eqn}
	C_{1}=\sqrt{ C_{z_{1}}^2+C_{u_{1}}^2 }=\sqrt{ \Czone^2+\Cuone^2 }=\Cone \ \dfrac{\text{м}}{\text{с}};
\end{equation}

Угол наклона вектора  для построения треугольников скоростей:
\begin{equation} \label{eu_eqn}
	a_{1}=arcctg\Bigl( \dfrac{C_{u_{1}}}{C_{z_{1}}} \Bigr)=arcctg\Bigl( \dfrac{\Cuone}{\Czone} \Bigr)=\aone \textdegree;
\end{equation}

Температура воздуха перед рабочим колесом:
\begin{equation} \label{eu_eqn}
	T_{1}=T_{1}^*-\dfrac{C_{1}^2}{2\cdot  \frac{k_{\text{в}}}{k_{\text{в}}-1}\cdot R_{\text{в}}}=\TN- \dfrac{\Cone^2}{2\cdot  \dfrac{\KN}{\KN-1}\cdot \RN}=\Too \ \text{К};
\end{equation}

Проекция относительной скорости W на окружное направление входной скорости на входе в рабочее колесо:
\begin{equation} \label{eu_eqn}
	W_{u_{1}}=C_{u_{1}}-u=\Cuone-\U=\Wuone \ \dfrac{\text{м}}{\text{с}};
\end{equation}

Относительная скорость на входе в колесо:
\begin{equation} \label{eu_eqn}
	W_{1}=\sqrt{ C_{z_{1}}^2+W_{u_{1}}^2 }=\sqrt{ \Czone^2+(\Wuone)^2 }=\Wone \ \dfrac{\text{м}}{\text{с}};
\end{equation}

Число Маха по относительной скорости на входе в рабочее колесо первой ступени:
\begin{equation} \label{eu_eqn}
	M_{W_{1}}=\dfrac{W_{1}}{\sqrt{ k_{\text{в}}\cdot R_{\text{в}}\cdot T_{1} }}= \dfrac{\Wone}{\sqrt{ \KN \cdot \RN \cdot \Too}}=\Mwone;
\end{equation}

Наклон входной относительной скорости при отсчете от отрицательного направления оси $u$ характеризуется углом $\beta$:
\begin{equation} \label{eu_eqn}
	\beta_{1}=arcctg\Bigl( \dfrac{W_{u_{1}}}{C_{z_{1}}} \Bigr)=arcctg\Bigl( \dfrac{\Wone}{\Czone} \Bigr)=\Bone \textdegree;
\end{equation}

Уменьшение осевой составляющей скорости в одной ступени:
\begin{equation} \label{eu_eqn}
	\Delta C_{z}=\dfrac{{C_{z_{1}}-C_{z_{2}}}}{i}=\dfrac{\Czone-\Cztwo}{15}=\DCz \ \dfrac{\text{м}}{\text{с}};
\end{equation}

Осевая составляющая скорости на выходе из рабочего колеса первой ступени:
\begin{equation} \label{eu_eqn}
	C_{z_{2}}=C_{z_{1}}- \dfrac{\Delta C_{z}}{2}=\Czone- \dfrac{\DCz}{2}=\OCztwo \ \dfrac{\text{м}}{\text{с}};
\end{equation}

Абсолютная скорость на выходе  в рабочее колесо:
\begin{equation} \label{eu_eqn}
	C_{2}=\sqrt{ C_{z_{2}}^2+C_{u_{2}}^2 }=\sqrt{\Cztwo^2 \cdot \Cutwo^2 }=\Cztwo\ \dfrac{\text{м}}{\text{с}};
\end{equation}

Угол наклона вектора  для построения треугольников скоростей:
\begin{equation} \label{eu_eqn}
	\alpha_{2}=arcctg\Bigl( \dfrac{C_{u_{2}}}{C_{z_{2}}} \Bigr)=arcctg\Bigl( \dfrac{\Cutwo}{\Cztwo} \Bigr)=\atwo \textdegree;
\end{equation}

Проекция относительной скорости $W$ на окружное направление входной скорости на выходе из рабочего колеса:
\begin{equation} \label{eu_eqn}
	W_{u_{2}}=C_{u_{2}}-u= \Cuone - \U=\Wutwo \ \dfrac{\text{м}}{\text{с}};
\end{equation}

Относительная скорость на выходе из колеса:
\begin{equation} \label{eu_eqn}
	W_{2}=\sqrt{ C_{z_{2}}^2+W_{u_{2}}^2 }=\sqrt{ \Cztwo ^2+(\Wutwo)^2 }=\Wtwo \ \dfrac{\text{м}}{\text{с}};
\end{equation}

Наклон выходной относительной скорости:
\begin{equation} \label{eu_eqn}
	\beta_{2}=arcctg\Bigl( \dfrac{W_{u_{2}}}{C_{z_{2}}} \Bigr)=arcctg\Bigl( \dfrac{\Wutwo}{\Cztwo} \Bigr)=\Btwo \textdegree;
\end{equation}

Угол поворота в решетке рабочего колеса:
\begin{equation} \label{eu_eqn}
	\varepsilon= \beta_{2}-\beta_{1}=\Btwo -\Bone =\Epsi \textdegree;
\end{equation}

Коэффициент расхода на внешнем диаметре:
\begin{equation} \label{eu_eqn}
	\varphi_{\text{н}}= \dfrac{C_{z_{1}}}{u_{\text{н}_{1}}}=\dfrac{\Czone}{\Unone}=\Phin;
\end{equation}

Проверка числа Маха по средней относительной скорости на внешнем диаметре первой ступени:
\begin{equation} \label{eu_eqn}
	M_{\text{wc}}=u_{\text{н}_{1}}\cdot \dfrac{\sqrt{ 1+\varphi_{\text{н}}^2 }}{\sqrt{ k_{\text{в}}\cdot R_{\text{в}}\cdot T_{1}^* }}= \Unone \cdot \dfrac{\sqrt{ 1+ \Phin ^2 }}{\sqrt{ \KN \cdot \RN \cdot \TN }}=\Mwc;
\end{equation}

\image{speed.png}{Треугольник скоростей на среднем диаметре первой ступени компрессора.}{1}

\newpage
\section{Расчет турбины}
\subsection{Исходные данные для расчета}

По методическим указаниям \cite{TURB} произведем расчет турбины. Из ранее полученных результатов мы получили основные значения для предварительного расчета турбины.

\begin{enumerate} 
  \item Полное давление и полная температура на входе в турбину
	\begin{equation} \label{eu_eqn}
		P_{0}^*=\sigma_{\text{кс}}^*\cdot P_{\text{к}},\text{МПа};
	\end{equation}
	
	где $p_{\text{к}}$ – давление на выходе из компрессора, найденное в Главе 3,
	$p_{\text{к}}^*=\Pk \ \text{МПа}$;
	$\sigma_{\text{кс}}$ – коэффициент потерь полного давления в камере сгорания, заданное при расчете в программе A2GTP, $\sigma_{\text{кс}}=\TSigks$;
	$$p_{0}^*=\TSigks \cdot \Pk=\TPo \ \text{МПа};$$
	$$T_{0}^* = \Ttri \ \text{К}.$$
  
  \item Рабочее тело – газ со следующими характеристиками:
	  \begin{itemize}
 
        \item Газовая постоянная - $R=\RNNN \ \dfrac{\text{Дж}}{\text{кг}\cdot \text{К}};$
        \item Показатель изоэнтропы - $k=\Kg;$
        \item Изобарная теплоёмкость при заданной температуре и давлении перед турбиной - $C_{p_{\text{г}}}=\Cpg \ \dfrac{\text{Дж}}{\text{кг}}\cdot \text{К}.$
 
      \end{itemize}
  \item Мощность проектируемой турбины
	\begin{equation} \label{eu_eqn}
		N_{\text{т}}=N_{\text{е}}+N_{\text{к}}=\N + \TNk = \TNt \ \text{МВт}
	\end{equation}

где $N_{\text{к}}$ – мощность, потребляемая компрессором		
	\begin{equation} \label{eu_eqn}
		N_{\text{к}}=H^*_{\text{к}}\cdot G_{\text{в}}=\Hkk \cdot 10^3 \cdot \GTN = \TNk \ \text{МВт}
	\end{equation}  
  
  \item Номинальный расход газа - $G_{\text{г}}=\GTN \ \dfrac{\text{кг}}{\text{с}}$; 
  \item Частота вращения турбины - $n=3000 \ \dfrac{\text{об}}{\text{мин}}$;
  \item Адиабатный КПД процесса расширени - $\eta_{\text{ад. т.}}=0.91$;
  \item Безразмерная скорость потока за турбиной - $\lambda_{c_{2}\text{т}}=\Lacm$;
  \item Угол выхода из последней ступени турбины - $a_{2\text{т}}=90\textdegree$;
  \item Коэффициент, учитывающий механические потери и потери от утечек рабочего тела - $k_{N}=\Tkn$.
\end{enumerate}

\subsection{Предварительный расчет турбины}

Удельная внутренняя мощность турбины
\begin{equation} \label{eu_eqn}
		H_{UT}=k_{N} \dfrac{N_{\text{т}}}{G_{\text{г}}}= 1.0185\cdot \dfrac{\TNt \cdot 10^3}{\GTN} =\THut \ \dfrac{\text{кДж}}{\text{кг}}
	\end{equation}

Температурный перепад на турбину по параметрам торможения
\begin{equation} \label{eu_eqn}
		\Delta T_{T}^*= \dfrac{H_{UT}}{C_{p_{\text{г}}}}= \dfrac{\THut}{\Cpg}= \TDTt \ \text{К}
	\end{equation}

Температура торможения за турбиной
\begin{equation} \label{eu_eqn}
		T_{2T}^*=T_{0}^*-\Delta T_{T}^*=\Ttri - \TDTt=\Ttwot \ \text{К}
	\end{equation}

Критическая скорость потока газа, выходящего из турбины
\begin{equation} \label{eu_eqn}
		a_{\text{кр}_{2}}=\sqrt{ \dfrac{2k}{k+1} RT_{2T}^*}=\sqrt{ \dfrac{2\cdot \Kg}{\Kg-1} \cdot \RNNN \cdot \Ttwot}= \akrtwo \ \dfrac{\text{м}}{\text{с}}
\end{equation}

Скорость потока газа за турбиной
\begin{equation} \label{eu_eqn}
		c_{2{\text{т}}}=\lambda_{c_{2} \text{т}}\cdot a_{\text{кр}_{2}}=\Lacm \cdot \akrtwo = \ctwom \ \dfrac{\text{м}}{\text{с}}
\end{equation}

Адиабатный перепад энтальпий на турбину
\begin{equation} \label{eu_eqn}
		H_{\text{ад.т}}= H_{UT}+ \dfrac{c_{2\text{т}}^2}{2}=\THut \cdot \dfrac{\ctwom^2}{2}=\Hadm \ \dfrac{\text{кДж}}{\text{кг}}
\end{equation}

Изоэнтропийный перепад энтальпий на турбину
\begin{equation} \label{eu_eqn}
		H_{\text{от}}=\dfrac{H_{\text{ад.т}}}{\eta_{\text{ад.т}}}=\dfrac{\Hadm}{0.91}=\Hot \ \dfrac{\text{кДж}}{\text{кг}}
\end{equation}

Температура в потоке за турбиной при изоэнтропийном процессе расширения
\begin{equation} \label{eu_eqn}
		T_{2t_{T}}^*=T_{0}- \dfrac{H_{\text{от}}}{C_{p_{{\text{г}}}}}=\Ttri-  \dfrac{\Hot \cdot 10^3}{\Cpg}=\Ttwott \ \text{К}
\end{equation}

Давление в потоке за турбиной
\begin{equation} \label{eu_eqn}
		p_{2T}=p_{0}^* \Bigl(\dfrac{T_{2t_{T}^*}}{T_{0}^*} \Bigr)^{\frac{k}{k-1}}=\TPo \cdot\Bigl(\dfrac{\Ttwott}{\Ttri} \Bigr)^{\frac{\Kg}{\Kg-1}}=\Ptwot \ \text{МПа}
\end{equation}

Температура в потоке за турбиной
\begin{equation} \label{eu_eqn}
		T_{2T}=T_{2T}^*- \dfrac{c_{2 \text{т}}^2}{2C_{p_{\text{г}}}}=\Ttwot - \dfrac{\ctwom^2}{2\cdot \Cpg}= \TtwoT \ \text{К}
\end{equation}

Плотность в потоке за турбиной
\begin{equation} \label{eu_eqn}
		\rho_{2T}=\dfrac{p_{2T}}{R \cdot T_{2T}}=\dfrac{\Ptwot \cdot 10^6}{\RNNN \cdot \Ttwott}= \Trhotwot \ \dfrac{\text{кг}}{\text{м}^3}
\end{equation}

Площадь сечения на выходе из рабочего колеса последней ступени
\begin{equation} \label{eu_eqn}
		F_{2T}=\dfrac{G_{\text{г}}}{\rho_{2T}\cdot c_{2\text{т}}\cdot \sin(a_{2\text{т}})}=\dfrac{\GTN}{\Trhotwot \cdot \ctwom \cdot 1}=\Ftwot \ \text{м}^2
\end{equation}

Напряжения в корневом сечении рабочей лопатки
\begin{equation} \label{eu_eqn}
		\sigma_{p}=0.89\cdot 10^{-5}\cdot n^2\cdot F_{2T}=0.89\cdot 10^{-5}\cdot 3000^2\cdot \Ftwot=\Sigp \ \text{МПа}
\end{equation}

Выберем материал для лопаток – сталь ЭИ696, для которой предел длительной прочности $[\sigma_{500}]=\SigSoo \ \text{МПа}$ и находим коэффициент запаса прочности
\begin{equation} \label{eu_eqn}
		K_{\text{пр}}=\dfrac{[\sigma_{500}]}{\sigma_{p}}=\dfrac{\SigSoo}{\Sigp}=\KPR,
\end{equation}

коэффициент запаса имеет значение в допустимых пределах $K_{\text{пр}}\geq 1.5$, т.е. условие прочности выполняется.

Далее следует выбрать средний диаметр. Его выбирают, ориентируясь на диаметральные габариты компрессора и камеры сгорания, и таким образом, чтобы окружная скорость на среднем диаметре не превышала $500 \ \dfrac{\text{м}}{\text{с}}$. Если она меньше $300 \ \dfrac{\text{м}}{\text{с}}$, то следует увеличить диаметр или частоту вращения ротора.

Для данного расчета примем $d_{2T}=\dtwom \ \text{м}$. Тогда окружная скорость на среднем диаметре рабочего колеса последней ступени:
\begin{equation} \label{eu_eqn}
		u_{2} =\dfrac{\pi \cdot d_{2T} \cdot n}{60}= \dfrac{3.14\cdot \dtwom \cdot 3000}{60}= \Tu \ \dfrac{\text{м}}{\text{с}}
\end{equation}

Высота лопаток последней ступени
\begin{equation} \label{eu_eqn}
	l_{2}=\dfrac{F_{2T}}{\pi d_{2T}}= \dfrac{\Ftwot}{3.14\cdot \dtwom }=\ltwo \ \text{м}
\end{equation}

Параметр:
\begin{equation} \label{eu_eqn}
	\dfrac{d_{2T}}{l_{2}}=\dfrac{\dtwom}{\ltwo}=\Totn
\end{equation}

Примем число ступеней турбины $m=4$. Тогда характерный напорный параметр $Y$ равен:
\begin{equation} \label{eu_eqn}
	Y=\dfrac{\sqrt{ \sum u_{2}^2 }}{\sqrt{ 2H_{\text{от}} }}=\dfrac{\sqrt{ 4\cdot \Tu^2}}{\sqrt{ 2\cdot \Hot}}=\Y,
\end{equation}

что соответствует рекомендованным значениям $(0,5...0,6)$.

\section{Заключение}

В данной работе проведен расчет параметров: тепловой расчет, расчет компрессорной части, расчет турбинной части.

В результате теплового расчета была выявлена оптимальная температура перед турбиной $T_3=1693 \ \text{К}$. Были получены оптимальные параметры $\eta_{\text{е}}=0.36$, $\pi_{\text{к}}^*=16$.

После проведения расчета был получен 15−ступенчатый компрессор со степенью сжатия $\pi_{\text{к}}^*=16$, габаритными параметрами $D_{\text{ср}_{1}}=\Dsrone \ \text{м}$, $D_{\text{н}_{1}}=\Dodi \ \text{м}$, $D_{\text{вт}}=\Dvtone \ \text{м}$. Для наглядности был построен треугольник скоростей
компрессора.

Подводя итог расчета турбинной части, была получения 4−ступенчатая турбина. Высота последней лопатки $l_{2}=177 \ \text{мм}$. Средний диаметр рабочих лопаток $d_{2T}=2.4 \ \text{м}$.