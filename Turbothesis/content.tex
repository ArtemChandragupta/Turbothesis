\sloppy
\section*{Введение}
\addcontentsline{toc}{section}{\protect{}Введение}
Принцип действия ГТУ сводится к следующему. Из атмосферы воздух забирают компрессором, после чего при повышенном давлении его подают в камеру сгорания, куда одновременно подводят жидкое топливо топливным насосом или газообразное топливо от газового компрессора. В камере сгорания воздух разделяется на два потока: один поток в количестве, необходимом для сгорания топлива, поступает внутрь жаровой трубы; второй – обтекает жаровую трубу снаружи и подмешивается к продуктам сгорания для понижения их температуры. Процесс сгорания в камере происходит при почти постоянном давлении.   Получающийся после смешения газ поступает в газовую турбину, в которой, расширяясь, совершает работу, а затем выбрасывается в атмосферу.   В отличие от паротурбинной установки полезная мощность ГТУ составляет только 30-50\% мощности турбины. Долю полезной мощности можно увеличить, повысив температуру газа перед турбиной или снизить температуру воздуха, засасываемого компрессором. В первом случае возрастает работа расширения газа в турбине, во втором – уменьшается работа, затрачиваемая на сжатие воздуха в компрессоре. Оба способа приводят к увеличению доли полезной мощности. Полезная мощность ГТУ также зависит от аэродинамических показателей проточных частей турбины и компрессора: чем меньше аэродинамические потери в турбине и компрессоре, тем большая доля мощности газовой турбины становится полезной.  Эффективность ГТУ в сравнении с другими тепловыми двигателями обнаруживается только при высокой температуре газа и высокой экономичности турбины и компрессора. Поэтому простой по принципу действия газотурбинный двигатель стали применяется в промышленности позднее других тепловых двигателей, т.е после того, как был достигнут прогресс в технологии получения жаропрочных материалов и накоплены необходимые знания в области аэродинамики турбомашин. \cite{PERV}

\newpage
\section{Термодинамический и газодинамический расчет}
\subsection{Исходные данные}

\begin{enumerate} 
  \item Полезная мощность $N=\N \ \text{МВт}$
  \item Температура газа перед турбиной $T_3^*=\Ttri\ \text{К}$
  \item Параметры наружного воздуха $P_H=\PN \ \text{МПа}; T_H=\TN \ \text{К}$
  \item Топливо – природный газ
  \item Прототип установки – ГТЭ-65
  \item Частота вращения вала – $n=\n\ \tfrac{\text{об}}{\text{мин}}$
\end{enumerate} 
Примем два упрощения при расчете в разделе 1:
\begin{enumerate} 
  \item Охлаждение турбины не учитывается, расход охладителя равен нулю;
  \item Не учитывается зависимость теплоемкости газа от температуры рабочего тела, принимается по рекомендациям пособия \cite{PERV}.
\end{enumerate}

\image{HeatScheme.jpg}{Тепловая схема одновальной ГТУ: К-компрессор; КС-камера сгорания; ГТ-газовая турбина.}{0.7}

\image{HeatGraph.png}{Цикл одновальной ГТУ простого типа в T-s–диаграмме: 1-2 – адиабатное сжатие в компрессоре, 2-3 – изобарный подвод теплоты в камере сгорания, 3-4 – адиабатное расширение продуктов сгорания на лопатках газовой турбины, 4-1 – изобарный отвод теплоты от продуктов сгорания в атмосферу.}{0.7}

\newpage
\subsection{Методы и пример расчета параметров рабочего процесса в характерных сечениях проточной части ГТУ. Определение основных характеристик ГТУ}

Расчет производится по методике из пособия \cite{PERV} (с.77-78)

Зададимся параметром степени повышения давления $\pi_{\text{к}}^*=\tfrac{P_{2}^*}{P_{1}^*}=\Pik$.

Газовая постоянная воздуха: $R_{\text{в}}=\RN \ \tfrac{\text{кДж}}{\text{кг}\cdot \text{К}}$

Удельная изобарная теплоёмкость воздуха: $c_{p_\text{в}}=\CPN \ \tfrac{\text{кДж}}{\text{кг}\cdot \text{К}}$

Коэффициент Пуассона воздуха: $k_{\text{в}}=\tfrac{c_{p_{\text{в}}}}{c_{p_{\text{в}}}-R_{\text{в}}}=\tfrac{\CPN}{\CPN-\RN}=\KN$

Принимаем коэффициент потерь полного давления во входном устройстве ГТУ $\sigma^*_{\text{вх}}=\SigEN$;

Давление воздуха перед компрессором:
\begin{equation} \label{}
  P_{1}^*=
  \sigma_{\text{вх}}^*\cdot P_{\text{н}}=
  \SigEN              \cdot \PN
  =\Podi \ \text{МПа};
\end{equation}

Температура воздуха перед компрессором:
\begin{equation} \label{}
  T_{1}^*=
    T_{\text{н}}
  =\TN \ \text{К};
\end{equation}

Давление воздуха за компрессором:
\begin{equation} \label{}
  P_{2}^*=
    \pi_{\text{к}}^*\cdot P_{1}^*=
    \Podi           \cdot \Pik
  =\Pdwa \ \text{МПа};
\end{equation}

Определим $T_2^*$ (температуру воздуха за компрессором):
\begin{equation} \label{}
  T_{2}^*=
    T_{\text{н}}^*\cdot \left(\pi_{\text{к}}^*\right)^{\tfrac{k_{\text{в}}-1}{k_{\text{в}}}}=
    \TN           \cdot      (\Pik                  )^{\tfrac{\KN         -1}{\KN         }}
  =\Tdwa \ \text{К};
\end{equation}

Работа, соответствующая изоэнтропийному перепаду в компрессоре:
\begin{equation} \label{}
  H_{\text{ок}}^*=
    C_{p_{\text{в}}}\cdot T_{1}^* \cdot \left[(\pi_{\text{к}}^*)^{\tfrac{k_{\text{в}}-1}{k_{\text{в}}}}-1 \right]=
    \CPN            \cdot \TN     \cdot \left[(\Pik            )^{\tfrac{\KN-1}{\KN}}-1                   \right]
  =\Hok \ \dfrac{\text{кДж}}{\text{кг}};
\end{equation}

$\eta_{\text{к ад}}=\EtaKAD$ - политропный к.п.д. компрессора, его выбор для расчета обусловлен тем, что он мало зависит от степени повышения давления в компрессоре $\pi^*_{\text{к}}$.

Полезная работа в компрессоре:
\begin{equation} \label{}
  H_{\text{к}}=
    \dfrac{H_{\text{ок}}^*}{\eta_{\text{к ад}}}=
    \dfrac{\Hok}{\EtaKAD}
  =\Hk \  \dfrac{\text{кДж}}{\text{кг}};
\end{equation}

Принимаем коэффициент потерь полного давления в камере сгорания $\sigma^*_{\text{кс}}=\SigBU$;

Давление газа перед турбиной:
\begin{equation} \label{}
  P_{3}^*=
    P_{2}^* \cdot \sigma^*_{\text{кс}}=
    \pi_{\text{к}}^* \cdot P_{1}^* \cdot \sigma^*_{\text{кс}}=
    \Pik             \cdot \Podi   \cdot \SigBU
  =\Ptri \ \text{МПа};
\end{equation}

Принимаем коэффициент потерь полного давления в выходном устройстве ГТУ $\sigma^*_{\text{вых}}=\SigOUT$;

Давление газа за турбиной:
\begin{equation} \label{}
  P_{4}^*=
    \dfrac{P^*_{\text{н}}}{\sigma^*_{\text{вых}}}=
    \dfrac{\PN}{\SigOUT}
  =\PN \ \text{МПа};
\end{equation}

Степень расширения газа в турбине:
\begin{equation} \label{}
  \pi_{\text{т}}^*=
    \dfrac{P_{3}^*}{P_{4}^*}=
    \dfrac{\Ptri}{\Pdwa}
  =\PiT;
\end{equation}

$k_{\text{г}}= \Kg$ – показатель изоэнтропы газа;

$R_{\text{г}}= \RN \ \tfrac{\text{кДж}}{\text{кг}\cdot \text{К}}$ – индивидуальная газовая постоянная;

$c_{p_{\text{г}}}=\CPg \ \tfrac{\text{кДж}}{\text{кг}\cdot \text{К}}$ – удельная изобарная теплоёмкость газа;

\

Работа, соответствующая изоэнтропийному перепаду в турбине:
\begin{multline} \label{}
  H_{\text{от}}^*=
    c_{p_{\text{г}}} \cdot T_{3}^* \cdot \left[1-(\pi_{\text{т}}^*)^{-\tfrac{k_{\text{г}}-1}{k_{\text{г}}}}\right]=\\
    =\CPg            \cdot \Ttri   \cdot \left[1-(\PiT            )^{-\tfrac{\Kg-1}{\Kg}}                  \right]
  =\Hot \ \dfrac{\text{кДж}}{\text{кг}};
\end{multline}

Принимаем политропный кпд турбины $\eta_{\text{т пол}}=\EtaTpol$

Полезная работа в турбине:
\begin{equation} \label{}
  H_{\text{т}}=
    H_{\text{от}}^* \cdot \eta_{\text{т пол}}=
    \Hot            \cdot \EtaTpol
  =\Ht \  \dfrac{\text{кДж}}{\text{кг}};
\end{equation}

Температура газа за турбиной $T_{4}^*$ определяется как:
\begin{equation} \label{}
  T_{4}^*=
    T_{3}^* \cdot (\pi_{\text{т}}^*)^{-\tfrac{k_{\text{г}}-1}{k_{\text{г}}}}=
    \Ttri   \cdot (\PiT)^{\tfrac{\Kg-1}{\Kg}}
  =\Tche \ \text{К};
\end{equation}

Примем коэффициенты механических потерь в турбине и компрессоре $\eta_{\text{мт}}=\Etamt$; $\eta_{\text{мк}}=\Etamk$;

Расход воздуха через компрессор:
\begin{equation} \label{}
  G_{\text{в}}=
    \dfrac{N_{e}}{H_{\text{т}}\cdot \eta_{\text{мт}}-\dfrac{H_{\text{к}}}{\eta_{\text{мк}}}}=
    \dfrac{\N\cdot 10^6}{\Ht \cdot 10^3 \cdot \Etamt-\dfrac{\Hk \cdot 10^3}{\Etamk}}
  =\GN \ \dfrac{\text{кг}}{\text{с}};
\end{equation}

Теплота с учетом потерь в камере сгорания:
\begin{equation} \label{}
  Q_{1}'=
    c_{p_\text{г}}\cdot(T_{3}^* - T_{2}^*)=
    \CPg          \cdot(\Ttri   - \Tdwa)
  =\Qii \ \dfrac{\text{кДж}}{\text{кг}};
\end{equation}

Примем КПД камеры сгорания $\eta_{\text{кс}}=\Etaks$;

Расход теплоты:
\begin{equation} \label{}
  Q_{1}=
    \dfrac{Q_{1}'}{\eta_{\text{кс}}}=
    \dfrac{\Qii}{\Etaks}
    =\Qodi \ \dfrac{\text{кДж}}{\text{кг}};
\end{equation}

Эффективный КПД ГТУ:
\begin{equation} \label{}
  \eta_{\text{е}}=
    \dfrac{H_{\text{т}}   \cdot \eta_{\text{мт}} - \dfrac{ H_{\text{к}} }{\eta_{\text{мк}}}}{Q_{1}           }=
    \dfrac{\Ht \cdot 10^3 \cdot \Etamt           - \dfrac{\Hk \cdot 10^3}{\Etamk          }}{\Qodi \cdot 10^3}
  =0.36;
\end{equation}

Коэффициент полезной работы ГТУ:
\begin{equation} \label{}
  \varphi=
    \dfrac{H_{\text{т}}\cdot \eta_{\text{мт}} -\dfrac{H_{\text{к}}}{\eta_{\text{мк}}}}{H_{\text{т}}\cdot \eta_{\text{мт}}}=
    \dfrac{\Ht         \cdot 10^3 \cdot \Etamt-\dfrac{\Hk\cdot 10^3}{\Etamk}}{\Ht \cdot 10^3       \cdot \Etamt}
  =\phi;
\end{equation}

Относительное количество воздуха, содержащегося в продуктах сгорания за камерой сгорания:
\begin{multline}
  g_{\text{в}}=
    \dfrac{Q_{\text{p}}^{\text{н}}\cdot \eta_{\text{кс}}+h_{\text{т}}+L_{0}\cdot c_{p_{\text{в}}} \cdot t_{2}^*- (L_{0}+1)\cdot (c_{p_{\text{г}}})_{\alpha=1}\cdot t_{3}}{c_{p_{\text{в}}}\cdot (t_{3}^*-t_{2}^*)}= \\
    =\dfrac{\Qnp \cdot \Etaks+\htop+\Lo\cdot \CPN \cdot \ttdwa- (\Lo+1)\cdot \Cpao \cdot \tttri}{\Cpao\cdot (\tttri-\ttdwa)}
  =\gv,
\end{multline}
\begin{eqexpl}
  \item{$Q_{\text{p}}^{\text{н}}$} низшая теплота сгорания топлива (принимаем $Q_{\text{p}}^{\text{н}}=\Qnp \ \tfrac{\text{кДж}}{\text{К}}$);
  \item{$h_{\text{т}}$} Энтальпия топлива (пренебрежимо мала);
  \item{$(c_{p_{\text{г}}})_{\alpha=1}$} Удельная теплоемкость газа при $\alpha=1$ \\ (принимаем $(c_{p_{\text{г}}})_{\alpha=1}=\Cpao\ \tfrac{\text{кДж}}{\text{К}}$);
  \item{$L_{0}$} Стехиометрический коэффициент (принимаем $L_{0}=15 \ \tfrac{\text{кг}_{\text{возд}}}{\text{кг}_{\text{топ}}}$).
\end{eqexpl}

Коэффициент избытка воздуха в камере сгорания
\begin{equation} \label{}
  \alpha=
    \dfrac{L_{0}+g_{\text{в}}}{L_{0}}=
    \dfrac{\Lo+\gv}{\Lo}
  =\aaa;
\end{equation}

Относительный расход топлива:
\begin{equation} \label{}
  g_{\text{т}}=
    \dfrac{1}{\alpha\cdot L_{0}}=
    \dfrac{1}{\aaa \cdot \Lo}
  =\gt;
\end{equation}

Допустимая температура для стали лопаток: $T_{\text{ст}}= \Tst\ \text{К};$

Расход воздуха для охлаждения статора:
\begin{equation} \label{}
  g_{\text{охл}}^{\text{с}}=
    0.01 + 0.25\cdot10^{-4} \cdot (T_{3}^* - T_{\text{ст}})=
    0.01 + 0.25\cdot10^{-4} \cdot (\Ttri   - \Tst         )
  =\gcohl;
\end{equation}

Расход воздуха для охлаждения ротора:
\begin{equation} \label{}
  g_{\text{охл}}^{\text{р}}=
    0.08 + 0.22\cdot10^{-4} \cdot (T_{3}^* - T_{\text{ст}})=
    0.08 + 0.22\cdot10^{-4} \cdot (\Ttri   - \Tst         )
  =\gpohl;
\end{equation}

Общий расход воздуха для охлаждения турбины:
\begin{equation} \label{}
  g_{\text{охл}}=
    \sigma_{\text{ут}}\cdot(g_{\text{охл}}^{\text{с}} + g_{\text{охл}}^{\text{р}})=
    \SigUT            \cdot(\gcohl                    + \gpohl)
  =\gohl;
\end{equation}

Относительный расход охлаждающего воздуха по отншению к расходу воздуха через компрессор:
\begin{equation} \label{}
  g_{\text{охл}}'=
    \dfrac{(1+g_{\text{т}})\cdot g_{\text{охл}}}{1+(1+g_{\text{т}})\cdot g_{\text{охл}}}=
    \dfrac{(1+\gt)         \cdot \gohl}{1+(1+\gt)\cdot \gohl}
  =\gohll;
\end{equation}

Расход топлива:
\begin{equation} \label{}
  G_{\text{т}}=
    g_{\text{т}}\cdot(1-g_{\text{охл}}')\cdot G_{\text{в}}=
    \gt         \cdot(1-\gohll         )\cdot \GN
  =\Gt\ \dfrac{\text{кг}}{\text{с}};
\end{equation}

Коэффициент располагаемой мощности:
\begin{equation} \label{}
  \Omega_{\text{рас}}=
    H_{\text{от}}^* \cdot \dfrac{G_{\text{в}}}{G_{\text{т}}}=
    \Hot            \cdot \dfrac{\GN}{\Gt}
  =\Omeras\ \dfrac{\text{кДж}}{\text{К}};
\end{equation}

Удельная эффективная работа ГТУ:
\begin{multline}
  H_{e}=
    (1+g_{\text{т}})\cdot (1-g_{\text{охл}}')\cdot H_{\text{Т}}\cdot \eta_{\text{тм}}-\dfrac{H_{\text{к}}}{\eta_{\text{мк}}}=\\
    =(1+\gt)\cdot (1-\gohll)\cdot \Ht \cdot \Etamt-\dfrac{\Hk}{\Etamk}
  =\He\ \dfrac{\text{кДж}}{\text{К}};
\end{multline}

Коэффициент полезной мощности:
\begin{equation} \label{}
  \Omega_{\text{пол}}=
    H_{e}^* \cdot \dfrac{{G_{\text{в}}}}{G_{\text{т}}}=
    \He     \cdot \dfrac{\GN}{\Gt}
  =\Omepol\ \dfrac{\text{кДж}}{\text{К}};
\end{equation}

\newpage
\section{Вариантный расчет ГТУ на ЭВМ}

Необходимо провести расчет параметров рабочего процесса в характерных сечениях проточной части и основных характеристик ГТУ при различных значениях степени повышения давления и температуры газа перед турбиной, по результатам расчета построить графики: $H_{e}, \eta_{e}, \varphi=f(\pi_{\text{К}}^*, T_3^*)$

\subsection{Результаты расчета}

Полные результаты рассчета смотреть в Приложении 1.

\inputpgf{Images/dgf}{KPD.pgf}{Зависимость эффективного КПД ГТУ от степени повышения давления в компрессоре, при различных значениях температуры.}

\inputpgf{Images/dgf}{Fi.pgf}{Зависимость коэффициента полезной работы ГТУ от степени повышения давления в компрессоре, при различных значениях температуры.}

\inputpgf{Images/dgf}{He.pgf}{Зависимость эффективной удельной работы ГТУ от степени повышения давления в компрессоре, при различных значениях температуры.}

\newpage
\subsection{Выбор степени повышения давления в компрессоре и начальной температуры газа перед турбиной}

Максимальный КПД установки достигается при максимальной температуре газа перед турбиной – $1743 \ \text{К}$. Жаростойкость материала лопаток турбины не позволяет выдерживать такую температуру, поэтому в качестве входной температуры на турбину выберем $\To \ \text{К}$. Экстремум графика зависимости эффективного КПД ГТУ от степени повышения давления в компрессоре наблюдается при $\pi_{\text{к}}^*=34$ и $\eta_e = 0.419$. Выбор такой степени сжатия неоправдан, т. к. при нём слишком низкие значения эффективной удельной работы и коэффициента полезной работы. Экстремум графика зависимости эффективной удельной работы ГТУ от степени повышения давления в компрессоре наблюдается при $\pi_{\text{к}}^*=16$, значение эффективного КПД ГТУ при этом $\eta_e = 0.314$. Коэффициент полезной работы ГТУ с увеличением  монотонно уменьшается, однако уменьшение  с целью увеличения  нецелесообразно, поскольку величина коэффициента полезной работы ГТУ увеличивается незначительно, при этом снижается величина эффективной удельной работы.

Таким образом, для дальнейших расчетов принимаем:

$T_3^* = \To \ \text{К}; \ \pi_{\text{к}}^*=16$

\newpage
\section{Приближенный расчет осевого компрессора}

Расчет производится в соответствии со схематическим продольным разрезом на рис.4.1. по методике из пособия \cite{COMP}.

\image{COMP-full.png}{Схема многоступенчатого осевого компрессора.}{1}

При приближенном расчете осевого компрессора основными расчетными сечениями являются: сечение 1-1 на входе в первую ступень и сечение 2-2  на выходе из последней ступени (рис.4.2). Определим параметры $P$ и $T$ в трех сечениях.

Давление воздуха в сечении 1-1:
\begin{equation} \label{}
  P_{1}^*=
    \sigma_{\text{вх}} \cdot P_{\text{н}}=
    \Sigin             \cdot \PN
  =\PCone \ \text{МПа},
\end{equation}
\begin{eqexpl}
  \item{$\sigma_{\text{вх}}^*$} коэффициент уменьшения полного давления во входной части компрессора (принимаем $\sigma_{\text{вх}}^*=\Sigin$).
\end{eqexpl}

Температура в сечении 1-1:
\begin{equation} \label{}
  T_{1}^*=T_{\text{н}}=\TN \ \text{К};
\end{equation}

Давление воздуха в сечении К-К:
\begin{equation} \label{}
  P_{\text{к}}^*=
    P_{\text{н}} \cdot \pi_{\text{к}}^*=
    \PN          \cdot \PiPi
  =\Pk \ \text{МПа},
\end{equation}
\begin{eqexpl}
  \item{$\pi_{\text{к}}^*$} степень повышения давления компрессоре (из первичного расчета $\pi_{\text{к}}^*=\PiPi$).
\end{eqexpl}

Давление в сечении 2-2:
\begin{equation} \label{}
  P_{2}^*=
    \dfrac{P_{\text{к}}^*}{\sigma_{\text{вых}}^*}=
    \dfrac{\Pk}{\Sigout}
  =\PCtwo \  \text{МПа},
\end{equation}
\begin{eqexpl}[10mm]
  \item{$\sigma_{\text{вых}}^*$} коэффициент увеличения давления в выходной части компрессора (принимаем $\sigma_{\text{вых}}^*=\Sigout$).
\end{eqexpl}

\image{COMP-lopatki.png}{Схема первой и последней ступеней компрессора.}{0.7}

Значение плотностей:
\begin{equation} \label{}
  \rho_{1}=
    \dfrac{P_{1}^*}{R_{\text{в}}\cdot T_{1}^*}=
    \dfrac{\PCone}{\RN \cdot \TN}
  =\Rhoone \ \dfrac{\text{кг}}{\text{м}^3};
\end{equation}

Примем КПД компресора $\eta^*_{\text{ад}}=\Etad$;

\begin{equation} \label{}
  \rho_{2}=\rho_{1}\left(\dfrac{P_{2}^*}{P_{1}^*}\right)^{\tfrac{1}{n}}=\Rhoone\left(\dfrac{\PCtwo}{\PCone}\right)^{\tfrac{1}{\nk}}=\Rhotwo \ \dfrac{\text{кг}}{\text{м}^3},
\end{equation}
\begin{eqexpl}
  \item{$n$} показатель политропы определяется из равенства:
\end{eqexpl}
\begin{equation} \label{}
  \dfrac{k}{k-1}\cdot \eta_{\text{ад}}^*=\dfrac{n}{n-1};
\end{equation}
$$\dfrac{\KN}{\KN-1} \cdot \Etad=\dfrac{n}{n-1} \Rightarrow n = \nk.$$

Примем величины осевой составляющей абсолютных скоростей в сечениях 1-1 и 2-2 соответственно $C_{z_1}=\Czone \  \tfrac{\text{м}}{\text{с}}$ и $C_{z_2}=\Cztwo \  \tfrac{\text{м}}{\text{с}}$. Втулочное отношение выберем $V_{1}=\tfrac{D_{\text{вт}_{1}}}{D_{\text{н}_{1}}}=\Vone$. Расход воздуха $G_{\text{в}}=\GNN \ \tfrac{\text{кг}}{\text{с}}$.

Из уравнения расхода первой ступени выразим значение наружного диаметра на входе в компрессор:
\begin{equation} \label{}
  G_{\text{в}}=\rho_{1}\cdot \dfrac{\pi}{4}\cdot \left(D_{\text{н}_{1}}^2\cdot D^2_{\text{вт}_{1}}\right)\cdot C_{z_{1}}=\rho_{1}\cdot D_{\text{н}_{1}}\cdot \left(1-\nu_{1}^2\right)\cdot C_{z_{1}};
\end{equation}

Откуда,
$$D_{\text{н}_{1}}=\sqrt{ \dfrac{4\cdot G_{\text{в}}}{\rho_{1}\cdot \pi\cdot \left(1-\nu_1^2 \right)\cdot C_{z_{1}}} }=\sqrt{ \dfrac{4\cdot \GNN}{\Rhoone \cdot 3.14\cdot (1-\Vone^2)\cdot \Czone} }=\Dodi \ \text{м};$$

Диаметр втулки первой ступени:
\begin{equation} \label{}
  D_{\text{вт}_{1}}=
    \nu_{1} \cdot D_{\text{н}_{1}}=
    \Vone   \cdot \Dodi
  =\Dvtone \ \text{м};
\end{equation}

Средний диаметр первой ступени:
\begin{equation} \label{}
  D_{\text{ср}_{1}}=
    \dfrac{D_{\text{н}_{1}} + D_{\text{вт}_{1}}}{2}=
    \dfrac{\Dodi            + \Dvtone}{2}
  =\Dsrone \ \text{м};
\end{equation}

Длина рабочей лопатки первой ступени:
\begin{equation} \label{}
  l_{1}=
    \dfrac{D_{\text{н}_{1}}-D_{\text{вт}_{1}}}{2}=
    \dfrac{\Dodi-\Dvtone}{2}
  =\Lone \ \text{м};
\end{equation}

Размеры проходного сечения 2-2:
\begin{equation} \label{}
  F_{2}=
    \dfrac{G_{\text{в}}}{C_{z_{2}}\cdot \rho_{2}}=
    \dfrac{\GNN}{\Cztwo \cdot \Rhotwo}
  =\Ftwo \ \text{м}^2;
\end{equation}

Принимаем в проточной части $D_{\text{вт}}=const$.

Тогда:
\begin{equation} \label{}
  \nu_{2}=
    \dfrac{1}{\sqrt{\dfrac{1+4F_{2}}{\pi\cdot D_{\text{вт}_{1}}^2} }}=
    \dfrac{1}{\sqrt{\dfrac{1+4\cdot \Ftwo}{\pi\cdot \Dvtone^2} }}
  =\Vtwo;
\end{equation}

Длина рабочей лопатки на последней ступени:
\begin{equation} \label{}
  l_{2}=
    \dfrac{1}{2}\left(\dfrac{1}{\nu_{2}}-1\right)\cdot D_{\text{вт}_{1}}=
    \dfrac{1}{2}\left(\dfrac{1}{\Vtwo}  -1\right)\cdot \Dvtone
  =\Ltwo \ \text{м};
\end{equation}

Для расчета частоты вращения необходимо задать окружную скорость на наружном диаметре первой ступени $u_{\text{н}_{1}}=\Unone \ \tfrac{\text{м}}{\text{с}}$, тогда:
\begin{equation} \label{}
  n=
    \dfrac{60\cdot u_{\text{н}_{1}}}{\pi\cdot D_{\text{н}_{1}}}=
    \dfrac{60\cdot \Unone}{3.14\cdot \Dodi}
  =\n \ \dfrac{\text{об}}{\text{мин}};
\end{equation}

Таким образом, для соединения вала турбоагрегата с валом генератора необходимо использовать редуктор, понижающий обороты до $3000\ \tfrac{\text{об}}{\text{мин}}$.

Адиабатический напор в проточной части компрессора по полным параметрам:
\begin{multline} \label{}
  H_{\text{ад. пр. ч.}}^*=
    \dfrac{k}{k-1}\cdot R_{\text{в}}\cdot T_{1}^*\cdot \left[\left(\dfrac{P_{2}^*}{P_{1}^*}\right)^{\tfrac{{k-1}}{k}}-1\right]=\\
    =\dfrac{\KN}{\KN-1}\cdot \RN \cdot \TN \cdot \left[\left(\dfrac{\PCtwo}{\PCone}\right)^{\tfrac{{\KN-1}}{\KN}}-1\right]
  =\Ha \ \dfrac{\text{кДж}}{\text{кг}};
\end{multline}

Приближенная величина теоретического напора или удельная работа, затрачиваемая на сжатие 1 кг воздуха:
\begin{equation} \label{}
  H_{\text{к}}^*=
    \dfrac{H_{\text{ад. пр. ч.}}^*}{\eta_{\text{ад}^*}}=
    \dfrac{\Ha}{\Etad}
  =\Hkk \ \dfrac{\text{кДж}}{\text{кг}};
\end{equation}

Выберем средний теоретический напор $h_{\text{ср}}=\hcp \ \tfrac{\text{кДж}}{\text{кг}}$.

Число ступеней компрессора:
\begin{equation} \label{}
  i=
    \dfrac{H^*_{\text{к}}}{h_{\text{ср}}}=
    \dfrac{\Hkk}{\hcp}
  =14.77;
\end{equation}

Принимаем $i=\i$.

Теоретический напор в первой ступени:
\begin{equation} \label{}
  h_{1}=(0.6\dots 0.7)\cdot h_{\text{ср}}=0.65\cdot \hcp=\hone \ \dfrac{\text{кДж}}{\text{кг}};
\end{equation}

Теоретический напор в средних ступенях:
\begin{equation} \label{}
  h_{\text{ср. ст.}}=(1.1\dots 1.2)\cdot h_{\text{ср}}=1.15 \cdot \hcp=\hm \ \dfrac{\text{кДж}}{\text{кг}};
\end{equation}

Теоретический напор в последней ступени:
\begin{equation} \label{}
  h_{\text{п}}=(0.95\dots 1)\cdot h_{\text{ср}}=1\cdot \hcp=\hcp \ \dfrac{\text{кДж}}{\text{кг}};
\end{equation}

\inputpgf{Images/dgf}{COMP-stup.pgf}{Распределение теоретического напора по ступеням компрессора.}

В результате распределения напоров соблюдается условие:
\begin{equation} \label{}
  \sum h_{i} \approx H^*_{\text{к}}=\Hkk \ \dfrac{\text{кДж}}{\text{кг}};
\end{equation}

Уточняем величину окружной скорости на среднем диаметре первой ступени:
\begin{equation} \label{}
  u_{\text{ср}_{1}}=
    \dfrac{\pi  \cdot D_{\text{ср}_{1}}\cdot n }{60}=
    \dfrac{3.14 \cdot \Dsrone          \cdot \n}{60}
  =\Usrone \ \dfrac{\text{м}}{\text{с}};
\end{equation}

Производим расчет первой ступени по среднему диаметру;

Коэффициент расхода на среднем диаметре:
\begin{equation} \label{}
  \varphi_{1}=
    \dfrac{C_{z_{1}}}{u_{\text{ср}_{1}}}=
    \dfrac{\Czone}{\Usrone}
  =\Phione;
\end{equation}

Коэффициент теоретического напора:
\begin{equation} \label{}
  \bar{h}_{1}=
    \dfrac{h_{1}}{u_{\text{ср}_{1}^2}}=
    \dfrac{\hone \cdot 10^3}{\Usrone^2}
  =\hhone;
\end{equation}

Отношение:
\begin{equation} \label{}
  \dfrac{\bar{h}_{1}}{\varphi}=\dfrac{\hhone}{\Phione}=\otn;
\end{equation}

Зададим степень реактивности $\Omega=\OME$ и найдем:
\begin{equation} \label{}
  \dfrac{\Omega}{\varphi}=\dfrac{\OME}{\Phione}=\otm;
\end{equation}

По графику 3.4 находим $\left(\dfrac{\bar{h}_{1}}{\varphi} \right)_{\tfrac{b}{t}=1}=\Pograph$.

\inputpgf{Images/dgf}{COMP-ome-approx.pgf}{График зависимости $\left(\tfrac{\bar{h}_{1}}{\varphi} \right)_{\tfrac{b}{t}=1}$ от $\tfrac{\Omega}{\varphi}$}

Коэффициент:
\begin{equation} \label{}
  J=
    \dfrac{\dfrac{\bar{h}_{1}}{\varphi}}{\left(\dfrac{\bar{h}_{1}}{\varphi} \right)_{\tfrac{b}{t}=1}}=
    \dfrac{\otn}{\Pograph}
  =\J;
\end{equation}

\inputpgf{Images/dgf}{COMP-J.pgf}{Изменение коэффициента $J$ в зависимости от густоты решетки.}

Пользуясь рисунком 3.5, определяем $\dfrac{b}{t}=\GraphJ \to \dfrac{t}{b}=\tb$.

При постоянной вдоль радиуса хорде относительный шаг у втулки первой ступени:
\begin{equation} \label{}
  \left(\dfrac{t}{b}\right)_{\text{вт}}=\dfrac{t}{b}\cdot \dfrac{D_{\text{вт}_{1}}}{D_{\text{ср}_{1}}}=\tb \cdot \dfrac{\Dvtone}{\Dsrone}=\tbem;
\end{equation}

Окружные скорости на входе и на выходе из рабочего колеса принимаем одинаковыми, т. е. $u_{\text{ср}_{1}}=u_{\text{ср}_{2}}=u=\U \ \tfrac{\text{м}}{\text{с}}$.

Проекция абсолютной скорости на окружное направление входной скорости на входе в рабочее колесо:
\begin{equation} \label{}
  C_{u_{1}}=
    u(1-\Omega)-\dfrac{h_{1}}{2u}=
    \U \cdot0.5- \dfrac{\hone \cdot 10^3}{2\cdot \U}
  =\Cuone\ \dfrac{\text{м}}{\text{с}};
\end{equation}

На выходе из рабочего колеса:
\begin{equation} \label{}
  C_{u_{2}}=
    u(1-\Omega)+\dfrac{h_{1}}{2u}=
    \U \cdot0.5+ \dfrac{\hone \cdot 10^3}{2\cdot \U}
  =\Cutwo \ \dfrac{\text{м}}{\text{с}};
\end{equation}

Абсолютная скорость на входе в рабочее колесо:
\begin{equation} \label{}
  C_{1}=
    \sqrt{ C_{z_{1}}^2+C_{u_{1}}^2 }=
    \sqrt{ \Czone^2+\Cuone^2 }
  =\Cone \ \dfrac{\text{м}}{\text{с}};
\end{equation}

Угол наклона вектора  для построения треугольников скоростей:
\begin{equation} \label{}
  a_{1}=
    arcctg\left( \dfrac{C_{u_{1}}}{C_{z_{1}}} \right)=
    arcctg\left( \dfrac{\Cuone}{\Czone}       \right)
  =\aone ^{\circ};
\end{equation}

Температура воздуха перед рабочим колесом:
\begin{equation} \label{}
  T_{1}=
    T_{1}^*-\dfrac{C_{1}^2}{2\cdot \dfrac{k_{\text{в}}}{k_{\text{в}}-1}\cdot R_{\text{в}}}=
    \TN    -\dfrac{\Cone^2}{2\cdot \dfrac{\KN}{\KN-1}                  \cdot \RN         }
  =\Too \ \text{К};
\end{equation}

Проекция относительной скорости W на окружное направление входной скорости на входе в рабочее колесо:
\begin{equation} \label{}
  W_{u_{1}}=
    C_{u_{1}} - u=
    \Cuone    - \U
  =\Wuone \ \dfrac{\text{м}}{\text{с}};
\end{equation}

Относительная скорость на входе в колесо:
\begin{equation} \label{}
  W_{1}=
    \sqrt{ C_{z_{1}}^2 + W_{u_{1}}^2 }=
    \sqrt{ \Czone   ^2 + (\Wuone) ^2 }
  =\Wone \ \dfrac{\text{м}}{\text{с}};
\end{equation}

Число Маха по относительной скорости на входе в рабочее колесо первой ступени:
\begin{equation} \label{}
  M_{W_{1}}=
    \dfrac{W_{1}}{\sqrt{ k_{\text{в}}\cdot R_{\text{в}}\cdot T_{1} }}=
    \dfrac{\Wone}{\sqrt{ \KN         \cdot \RN         \cdot \Too  }}
  =\Mwone;
\end{equation}

Наклон входной относительной скорости при отсчете от отрицательного направления оси $u$ характеризуется углом $\beta$:
\begin{equation} \label{}
  \beta_{1}=
    arcctg\left( \dfrac{W_{u_{1}}}{C_{z_{1}}} \right)=
    arcctg\left( \dfrac{\Wone}{\Czone}        \right)
  =\Bone ^{\circ};
\end{equation}

Уменьшение осевой составляющей скорости в одной ступени:
\begin{equation} \label{}
  \Delta C_{z}=
    \dfrac{{C_{z_{1}} - C_{z_{2}}}}{i}=
    \dfrac{\Czone     - \Cztwo}{\i}
  =\DCz \ \dfrac{\text{м}}{\text{с}};
\end{equation}

Осевая составляющая скорости на выходе из рабочего колеса первой ступени:
\begin{equation} \label{}
  C_{z_{2}}=
    C_{z_{1}} - \dfrac{\Delta C_{z}}{2}=
    \Czone    - \dfrac{\DCz}{2}
  =\OCztwo \ \dfrac{\text{м}}{\text{с}};
\end{equation}

Абсолютная скорость на выходе  в рабочее колесо:
\begin{equation} \label{}
  C_{2}=
    \sqrt{ C_{z_{2}}^2 + C_{u_{2}}^2 }=
    \sqrt{ \Cztwo   ^2 + \Cutwo   ^2 }
  =\Ctwo\ \dfrac{\text{м}}{\text{с}};
\end{equation}

Угол наклона вектора  для построения треугольников скоростей:
\begin{equation} \label{}
  \alpha_{2}=
    arcctg\left( \dfrac{C_{u_{2}}}{C_{z_{2}}} \right)=
    arcctg\left( \dfrac{\Cutwo}{\Cztwo}       \right)
  =\atwo ^{\circ};
\end{equation}

Проекция относительной скорости $W$ на окружное направление входной скорости на выходе из рабочего колеса:
\begin{equation} \label{}
  W_{u_{2}}=
    C_{u_{2}}-u=
    \Cuone - \U
  =\Wutwo \ \dfrac{\text{м}}{\text{с}};
\end{equation}

Относительная скорость на выходе из колеса:
\begin{equation} \label{}
  W_{2}=
    \sqrt{ C_{z_{2}}^2 + W_{u_{2}}^2 }=
    \sqrt{ \Cztwo   ^2 + (\Wutwo) ^2 }
  =\Wtwo \ \dfrac{\text{м}}{\text{с}};
\end{equation}

Наклон выходной относительной скорости:
\begin{equation} \label{}
  \beta_{2}=
    arcctg\left( \dfrac{W_{u_{2}}}{C_{z_{2}}} \right)=
    arcctg\left( \dfrac{\Wutwo}{\Cztwo}       \right)
  =\Btwo ^{\circ};
\end{equation}

Угол поворота в решетке рабочего колеса:
\begin{equation} \label{}
  \varepsilon=
    \beta_{2} - \beta_{1}=
    \Btwo     - \Bone
  =\Epsi ^{\circ};
\end{equation}

Коэффициент расхода на внешнем диаметре:
\begin{equation} \label{}
  \varphi_{\text{н}}=
    \dfrac{C_{z_{1}}}{u_{\text{н}_{1}}}=
    \dfrac{\Czone}{\Unone}
  =\Phin;
\end{equation}

Проверка числа Маха по средней относительной скорости на внешнем диаметре первой ступени:
\begin{equation} \label{}
  M_{\text{wc}}=
    u_{\text{н}_{1}}\cdot \dfrac{\sqrt{1+\varphi_{\text{н}}^2 }}{\sqrt{k_{\text{в}}\cdot R_{\text{в}}\cdot T_{1}^* }}=
    \Unone          \cdot \dfrac{\sqrt{1+\Phin             ^2 }}{\sqrt{\KN         \cdot \RN         \cdot \TN     }}
  =\Mwc;
\end{equation}

\inputpgf{Images/dgf}{speed.pgf}{Треугольник скоростей на среднем диаметре первой ступени компрессора.}

\newpage
\section{Расчет турбины}
\subsection{Исходные данные для расчета}

По методическим указаниям \cite{TURB} произведем расчет турбины. Из ранее полученных результатов мы получили основные значения для предварительного расчета турбины.

\begin{enumerate} 
  \item Полное давление и полная температура на входе в турбину:
    \begin{equation} \label{}
      P_{0}^*=\sigma_{\text{кс}}^*\cdot P_{\text{к}},\text{МПа},
    \end{equation}
    \begin{eqexpl}
      \item{$P_{\text{к}}$} давление на выходе из компрессора (найдено в разделе 2,\\ $P_{\text{к}}^*=\Pk \ \text{МПа}$);
      \item{$\sigma_{\text{кс}}$} коэффициент потерь полного давления в камере сгорания, заданное при расчете в программе A2GTP ($\sigma_{\text{кс}}=\TSigks$);
    \end{eqexpl}
    $$P_{0}^*=\TSigks \cdot \Pk=\TPo \ \text{МПа};$$
    $$T_{0}^* = \To \ \text{К}.$$
  \item Рабочее тело – газ со следующими характеристиками \cite{LPI}:
    \begin{itemize}
      \item Газовая постоянная - $R=\RNNN \ \tfrac{\text{Дж}}{\text{кг}\cdot \text{К}};$
      \item Показатель изоэнтропы - $k=\Kg;$
      \item Изобарная теплоёмкость при заданной температуре и давлении перед турбиной - $C_{p_{\text{г}}}=\Cpg \ \tfrac{\text{Дж}}{\text{кг}\cdot \text{К}}.$
    \end{itemize}
  \item Мощность проектируемой турбины $N_{\text{т}}$:
    \begin{equation} \label{}
      N_{\text{т}}=N_{\text{е}}+N_{\text{к}}=\N + \TNk = \TNt \ \text{МВт},
    \end{equation}
    \begin{eqexpl}
      \item{$N_{\text{к}}$} мощность, потребляемая компрессором:
    \end{eqexpl}
    \begin{equation} \label{}
      N_{\text{к}}=H^*_{\text{к}}\cdot G_{\text{в}}=\Hkk \cdot 10^3 \cdot \GTN = \TNk \ \text{МВт}.
    \end{equation}  
  \item Номинальный расход газа - $G_{\text{г}}=\GTN \ \tfrac{\text{кг}}{\text{с}}$; 
  \item Частота вращения турбины - $n=\n \ \tfrac{\text{об}}{\text{мин}}$;
  \item Адиабатный КПД процесса расширени - $\eta_{\text{ад. т.}}=0.91$;
  \item Безразмерная скорость потока за турбиной - $\lambda_{c_{2}\text{т}}=\Lacm$;
  \item Угол выхода из последней ступени турбины - $a_{2\text{т}}=90^{\circ}$;
  \item Коэффициент, учитывающий механические потери и потери от утечек рабочего тела - $k_{N}=\Tkn$.
\end{enumerate}

\subsection{Предварительный расчет турбины}

Удельная внутренняя мощность турбины:
\begin{equation} \label{}
  H_{UT}=
    k_{N}      \dfrac{N_{\text{т}}}{G_{\text{г}}}=
    \Tkn \cdot \dfrac{\TNt \cdot 10^3}{\GTN}
  =\THut \ \dfrac{\text{кДж}}{\text{кг}};
\end{equation}

Температурный перепад на турбину по параметрам торможения:
\begin{equation} \label{}
  \Delta T_{T}^*=
    \dfrac{H_{UT}}{c_{p_{\text{г}}}}=
    \dfrac{\THut \cdot 10^3}{\Cpg}
  =\TDTt \ \text{К};
\end{equation}

Температура торможения за турбиной:
\begin{equation} \label{}
  T_{2T}^*=
    T_{0}^* - \Delta T_{T}^*=
    \To     - \TDTt
  =\Ttwot \ \text{К};
\end{equation}

Критическая скорость потока газа, выходящего из турбины:
\begin{equation} \label{}
  a_{\text{кр}_{2}}=
    \sqrt{ \dfrac{2k}{k+1} RT_{2T}^*}=
    \sqrt{ \dfrac{2\cdot \Kg}{\Kg+1} \cdot \RNNN \cdot \Ttwot}
  =\akrtwo \ \dfrac{\text{м}}{\text{с}};
\end{equation}

Скорость потока газа за турбиной:
\begin{equation} \label{}
  C_{2{\text{т}}}=
    \lambda_{c_{2} \text{т}}\cdot a_{\text{кр}_{2}}=
    \Lacm                   \cdot \akrtwo 
  = \ctwom \ \dfrac{\text{м}}{\text{с}};
\end{equation}

Адиабатный перепад энтальпий на турбину:
\begin{equation} \label{}
  H_{\text{ад.т}}=
    H_{UT}           + \dfrac{C_{2\text{т}}^2}{2}=
    \THut \cdot 10^3 + \dfrac{\ctwom^2}{2}
  =\Hadm \ \dfrac{\text{кДж}}{\text{кг}};
\end{equation}

Изоэнтропийный перепад энтальпий на турбину:
\begin{equation} \label{}
  H_{\text{от}}=
    \dfrac{H_{\text{ад.т}}}{\eta_{\text{ад.т}}}=
    \dfrac{\Hadm}{0.91}
  =\Hott \ \dfrac{\text{кДж}}{\text{кг}};
\end{equation}

Температура в потоке за турбиной при изоэнтропийном процессе расширения:
\begin{equation} \label{}
  T_{2t_{T}}^*=
    T_{0}- \dfrac{H_{\text{от}}}{c_{p_{{\text{г}}}}}=
    \To  - \dfrac{\Hott \cdot 10^3}{\Cpg}
  =\Ttwott \ \text{К};
\end{equation}

Давление в потоке за турбиной:
\begin{equation} \label{}
  P_{2T}=
    P_{0}^* \left(\dfrac{T_{2t_{T}^*}}{T_{0}^*} \right)^{\tfrac{k}{k-1}}=
    \TPo \cdot\left(\dfrac{\Ttwott}{\Ttri} \right)^{\tfrac{\Kg}{\Kg-1}}
  =\Ptwot \ \text{МПа};
\end{equation}

Температура в потоке за турбиной:
\begin{equation} \label{}
  T_{2T}=
    T_{2T}^*- \dfrac{C_{2 \text{т}}^2}{2c_{p_{\text{г}}}}=
    \Ttwot  - \dfrac{\ctwom^2}{2\cdot \Cpg}
  =\TtwoT \ \text{К};
\end{equation}

Плотность в потоке за турбиной:
\begin{equation} \label{}
  \rho_{2T}=
    \dfrac{P_{2T}}{R \cdot T_{2T}}=
    \dfrac{\Ptwot    \cdot 10^6}{\RNNN \cdot \Ttwott}
  =\Trhotwot \ \dfrac{\text{кг}}{\text{м}^3};
\end{equation}

Площадь сечения на выходе из рабочего колеса последней ступени:
\begin{equation} \label{}
  F_{2T}=
    \dfrac{G_{\text{г}}}{\rho_{2T}\cdot C_{2\text{т}} \cdot \sin(a_{2\text{т}})}=
    \dfrac{\GTN}{\Trhotwot        \cdot \ctwom        \cdot 1}
  =\Ftwot \ \text{м}^2;
\end{equation}

Напряжения в корневом сечении рабочей лопатки:
\begin{equation} \label{}
  \sigma_{p}=
    0.89\cdot 10^{-5} \cdot n^2  \cdot F_{2T}=
    0.89\cdot 10^{-5} \cdot \n^2 \cdot \Ftwot
  =\Sigp \ \text{МПа};
\end{equation}

Выберем материал для лопаток – сталь ЭИ437Б \cite{STEL}, для которой предел длительной прочности $[\sigma_{500}]=\SigSoo \ \text{МПа}$ и находим коэффициент запаса прочности:
\begin{equation} \label{}
  K_{\text{пр}}=
    \dfrac{[\sigma_{500}]}{\sigma_{p}}=
    \dfrac{\SigSoo}{\Sigp}
  =\KPR,
\end{equation}

коэффициент запаса имеет значение в допустимых пределах $K_{\text{пр}}\geq 1.5$, т.е. условие прочности выполняется.

Далее следует выбрать средний диаметр. Его выбирают, ориентируясь на диаметральные габариты компрессора и камеры сгорания, и таким образом, чтобы окружная скорость на среднем диаметре не превышала $500 \ \tfrac{\text{м}}{\text{с}}$. Если она меньше $300 \ \tfrac{\text{м}}{\text{с}}$, то следует увеличить диаметр или частоту вращения ротора.

Для данного расчета примем $d_{2T}=\dtwom \ \text{м}$. Тогда окружная скорость на среднем диаметре рабочего колеса последней ступени:
\begin{equation} \label{}
  u_{2}=
    \dfrac{\pi \cdot d_{2T} \cdot n }{60}=
    \dfrac{3.14\cdot \dtwom \cdot \n}{60}
    =\Tu \ \dfrac{\text{м}}{\text{с}};
\end{equation}

Высота лопаток последней ступени:
\begin{equation} \label{}
  l_{2}=
    \dfrac{F_{2T}}{\pi        d_{2T} }=
    \dfrac{\Ftwot}{3.14 \cdot \dtwom }
  =\ltwo \ \text{м},
\end{equation}

в результате чего параметр $\dfrac{d_{2T}}{l_{2}}$:
\begin{equation} \label{}
  \dfrac{d_{2T}}{l_{2}}=\dfrac{\dtwom}{\ltwo}=\Totn.
\end{equation}

Примем число ступеней турбины $m=4$. Тогда характерный напорный параметр $Y$ равен:
\begin{equation} \label{}
  Y=
    \dfrac{\sqrt{ \sum u_{2}^2} }{\sqrt{ 2H_{\text{от}} }}=
    \dfrac{\sqrt{ 4\cdot \Tu^2} }{\sqrt{ 2\cdot \Hot    }}
  =\Y,
\end{equation}

что соответствует рекомендованным значениям $(0,5 \dots 0,6)$.

\newpage
\section*{Заключение}
\addcontentsline{toc}{section}{\protect{}Заключение}

В данной работе проведен расчет параметров: тепловой расчет, расчет компрессорной части, расчет турбинной части.

В результате теплового расчета была выявлена оптимальная температура перед турбиной $T_3= \To \ \text{К}$. Были получены оптимальные параметры $\eta_{\text{е}}=0.36$, $\pi_{\text{к}}^*= \PiPi$.

После проведения расчета был получен 15−ступенчатый компрессор со степенью сжатия $\pi_{\text{к}}^*= \PiPi$, габаритными параметрами $D_{\text{ср}_{1}}=\Dsrone \ \text{м}$, $D_{\text{н}_{1}}=\Dodi \ \text{м}$, $D_{\text{вт}}=\Dvtone \ \text{м}$. Для наглядности был построен треугольник скоростей
компрессора.

Подводя итог расчета турбинной части, была получения 4−ступенчатая турбина. Высота последней лопатки $l_{2}=177 \ \text{мм}$. Средний диаметр рабочих лопаток $d_{2T}= \dtwom  \ \text{м}$.
